% Prelude Part One

\part*{Prelude}

\section*{FOREWORD}

This APRS Protocol Reference document represents the coming-of-age of WB4APR’s baby.
Starting with a simple concept — a way to track the location of moving objects via packet radio
— programs using the APRS protocol have grown into perhaps the most popular packet radio
application in use today. It’s also become one of the most complex; from the simple idea grew,
and still grows, a tactical communications system of tremendous capability. Like many ham
projects, the APRS protocol was designed as it was being implemented, and many of its
intricacies have never been documented.

Until now. This specification defines the APRS on-air protocol with a precision and clarity that
make it a model for future efforts. The work done by members of the APRS Working Group, as
well as Technical Editor Ian Wade, G3NRW, should be recognized as a tremendous contribution
to the packet radio art. With this document available, there is now no excuse for any developer to
improperly implement the APRS protocol.

As an APRS Working Group member whose role was mainly that of observer, I was fascinated
with the interplay among the APRS authors and the Technical Editor as the specification took
form. Putting onto paper details that previously existed only in the minds of the authors exposed
ambiguities, unconsidered consequences, and even errors in what the authors thought they knew.
The discussion that followed each draft, and the questions Ian posed as he tried to wring out the
uncertainties, gave everyone a better understanding of the protocol. I am sure that this process has
already contributed to better interoperability among existing APRS applications.
Everyone who has watched the specification develop, from the initial mention in April 1999 until
release of this Version 1.0 document in August 2000, knows that the process took much longer
than was hoped. At the same time, they saw the draft transformed from a skeleton into a hefty
book of over 110 pages. With the specification now in hand, I think we can all say the wait was
worth it. Congratulations to the APRS Working Group and, in particular, to G3NRW, for a major
contribution to the literature of packet radio.

\begin{verbatim}
John Ackermann, N8UR
TAPR Vice President and APRS Working Group Administrative Chair
August 2000
\end{verbatim}


% TOC to follow
