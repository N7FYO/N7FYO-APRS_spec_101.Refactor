\section*{Preamble}

\subsection*{APRS Working Group}

The APRS Working Group is an unincorporated association whose members
undertake to further the use and enhance the value of the APRS protocols by
(a) publishing and maintaining a formal APRS Protocol Specification; (b)
publishing validation tests and other tools to enable compliance with the
Specification; (c) supporting an APRS Certification program; and (d) generally
working to improve the capabilities of APRS within the amateur radio
community.

Although the Working Group may receive support from TAPR and other
organizations, it is an independent body and is not affiliated with any
organization. The Group has no budget, collects no dues, and owns no assets.
The current members of the APRS Working Group are:

\begin{itemize}
\item John Ackermann, N8UR Administrative Chair \& TAPR Representative
\item Bob Bruninga, WB4APR Technical Chair, founder of APRS
\item Brent Hildebrand, KH2Z Author of APRS+SA
\item Stan Horzepa, WA1LOU Secretary
\item Mike Musick, N0QBF Author of pocketAPRS
\item Keith Sproul, WU2Z Co-Author of WinAPRS/MacAPRS/X-APRS
\item Mark Sproul, KB2ICI Co-Author of WinAPRS/MacAPRS/X-APRS
\end{itemize}

\section*{Acknowledgements}
This document is the result of contributions from many people. It includes
much of the material produced by individual members of the Working
Group.

In addition, the paper on the Mic-E data format by Alan Crosswell, N2YGK,
and Ron Parsons, W5RKN was a useful starting point for explaining the
complications of this format.

\section*{Document Version Number}

Except for the very first public draft release of the APRS Protocol
Reference, the document version number is a 3-part number “P.p.D” (for
an approved document release) or a 4-part number “P.p.Dd” (for a draft
release):


% Figure/Table  Here


Thus, for example:

\begin{itemize}

\item Document version number “1.2.3” refers to document release 3 covering
APRS Protocol Version 1.2.

\item Document version number “1.2.3c” is draft “c” of that document.

\end{itemize}


\section*{Release History}

The release history for this document is listed in Appendix 7.

\section*{Document Conventions}

This document uses the following conventions:
\begin{itemize}

\item \texttt {Courier font} ASCII characters in APRS data.

\item \textvisiblespace  ASCII space character.

\item … (ellipsis) zero or more characters.

\item /\$ Symbol from Primary Symbol Table.

\item \textbackslash\$ Symbol from Alternate Symbol Table.

\item 0x hexadecimal (e.g. 0x1d).

\item All callsigns are assumed to have SSID –0 unless otherwise specified.

\item \hl{Yellow marker} (appears as light gray background in hard copy).
Marks text of interest — especially useful for highlighting single
literal ASCII characters (e.g. ") where they appear in APRS data.

\item Shaded areas in packet format diagrams are optional fields.

\end{itemize}

\section*{Feedback}

Please address your feedback or other comments regarding this document to
the TAPR aprsspec mail list.

To join the list, start at http://www.tapr.org and then follow the path Special
Interest Groups $\Rightarrow$ APRS Specification $\Rightarrow$ Join APRS Spec Discussion List.


\section*{Authors’ Foreword}

This reference document describes what is known as APRS Protocol
Version 1.0, and is essentially a description of how APRS operates
today.  It is intended primarily for the programmer who wishes to
develop APRS compliant applications, but will also be of interest to
the ordinary user who wants to know more about what goes on “under the
hood”.  It is not intended, however, to be a dry-as-dust, pedantic,
RFC-style programming specification, to be read and understood only by
the Mr Spocks of this world. We have included many items of general
information which, although strictly not part of the formal protocol
description, provide a useful background on how APRS is actually used
on the air, and how it is implemented in APRS software. We hope this
will put APRS into perspective, will make the document more readable,
and will not offend the purists too much.

It is important to realize how APRS originated, and to understand the
design philosophy behind it. In particular, we feel strongly that APRS
is, and should remain, a light-weight tactical system — almost anyone
should be able to use it in temporary situations (such as emergencies
or mobile work or weather watching) with the minimum of training and
equipment.

This document is the result of inputs from many people, and collated
and massaged by the APRS Working Group. Our sincere thanks go to
everyone who has contributed in putting it together and getting it
onto the street. If you discover any errors or omissions or misleading
statements, please let us know — the best way to do this is via the
TAPR aprsspec mailing list at www.tapr.org.

Finally, users throughout the world are continually coming up with new
ideas and suggestions for extending and improving APRS. We welcome
them.  Again, the best way to discuss these is via the aprsspec list.

--The APRS Working Group, August 2000


\section*{Disclaimer}

Like any navigation system, APRS is not infallible. No one should rely
blindly on APRS for navigation, or in life-and-death situations. Similarly,
this specification is not infallible.

The members of the APRS Working Group have done their best to define the
APRS protocol, but this protocol description may contain errors, or there
may be omissions. It is very likely that not all APRS implementations will
fully or correctly implement this specification, either today or in the future.


We urge anyone using or writing a program that implements this
specification to exercise caution and good judgement. The APRS Working
Group and the specification’s Editor disclaim all liability for injury to
persons or property that may result from the use of this specification or
software implementing it.

