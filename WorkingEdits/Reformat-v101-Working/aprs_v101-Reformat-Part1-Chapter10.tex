\chapter{Chapter 10: Mic-E Data Format}

Example of
Decoding the
Information Field
Data

If the first 9 bytes of the Information field contain ‘(_f n "Oj/, and the
destination address specifies that the station is in the western hemisphere
with a longitude offset of +100 degrees, then the data is decoded as follows:

\begin{itemize}
\item  ‘ is the APRS Data Type Identifier for a Mic-E packet containing current GPS data.
\item ( is the d+28 byte. The ( character has the value 40 decimal. Subtracting
28 gives 12. The longitude offset (in the destination address) is +100
degrees, so the longitude is 100 + 12 = 112 degrees.

\item _ is the m+28 byte. The _ character has the value 95 decimal. Subtracting
28 gives 67. This is ™ 60, so subtracting 60 gives a value of 7 minutes

\item longitude.

\item f is the h+28 byte. The f character has the value 102 decimal.

\item Subtracting 28 gives 74 hundredths of a minute.
Thus the longitude is 112 degrees 7.74 minutes west.
The speed and course are calculated as follows:

\item n is the SP+28 byte. The n character has the value 110 decimal. After
subtracting 28, the result is 82. As this is ™ 80, a further 80 is subtracted,
leaving a result of 2 tens of knots.

\item " is the DC+28 byte. The " character has the value 34 decimal.

\item Subtracting 28 gives 6. Dividing this by 10 gives a quotient of 0 (units of
speed). Adding the first part of the speed, multiplied by 10 (i.e. 20) to the
quotient (0) gives a final computed speed of 20 knots.
The remainder from the division is 6. Subtracting 4 gives the course in
hundreds of degrees; i.e. 2.

\item O (upper-case letter “O”) is the SE+28 byte. The O character has the value
79 decimal. Subtracting 28 gives 51. Adding this to the remainder
calculated above, multiplied by 100 (i.e. 200), gives the final value of
251 degrees for the course.

\end {itemize}

The last two characters (j/) represent the jeep symbol from the Primary
Symbol Table.



\section {Mic-E Position Ambiguity}

As mentioned in Chapter 6 (Time and Position Formats), a station may
reduce the precision of its position by introducing position ambiguity. This is
also possible in Mic-E data format.

The position ambiguity is specified for the latitude (in the destination
address). The same degree of ambiguity will then also apply to the longitude.
For example, if the destination address is T4SQZZ, the last two digits of the
latitude are ambiguous (represented by ZZ). Then, if the longitude data in the
Information field is (_f , as in the above example, the last two digits of the
computed longitude will be ignored — that is, the longitude will be 112
degrees 7 minutes.

\section {Mic-E Telemetry Data}

The Information field may optionally contain either Mic-E telemetry data
values or Mic-E status text.
If the byte following the Symbol Table Identifier is one of the Telemetry
Flag characters (‘,' or 0x1d), then telemetry data follows:
Optional Mic-E Telemetry Data
Telemetry
Flag

Bytes:

Telemetry Data Channels

F

Ch 1

Ch 2

Ch 3

Ch 4

Ch 5

1

1/2

1/2

1/2

1/2

1/2


The Telemetry Flag F is one of:

\begin{itemize}
  
\item ‘
2 printable hex telemetry values follow (channels 1 and 3).
'
5 printable hex telemetry values follow.
0x1d 5 binary telemetry values follow (Rev. 0 beta units only).
If F is ‘ or ', each channel requires 2 bytes, containing a 2-digit printable
hexadecimal representation of a value ranging from 0–255. For example, 254
is represented as FE.
If F is 0x1d, each channel requires one byte, containing an 8-bit binary value.
For example, if the telemetry data is '7200007100, the ' indicates that 5
bytes of telemetry follow, coded in hexadecimal:
0x72 = 114 decimal
0x00 = 0 decimal
0x00 = 0 decimal
0x71 = 113 decimal
0x00 = 0 decimal

\end {itemize}

\subsection{Mic-E Status Text}

As an alternative to telemetry data, the packet may include Mic-E status text.
The status text may be any length that fits in the rest of the Information field.
The Mic-E status text must not start with ‘,' or 0x1d, otherwise it will be
confused with telemetry data.

It is possible to include a standard APRS-formatted position in the Mic-E
status text field. A suitable position will cause the APRS display software to
override any position data the Mic-E has encoded. This is useful if using a
Mic-E without a GPS receiver.


Note: The Kenwood radios automatically insert a special type code at the
front of the status text string (i.e. in the 10th character of the Information
field):
>
Kenwood TH-D7:
Kenwood TM-D700: ]

These characters should not be confused with the APRS Data Type Identifier
that appears at the start of reports.

It is envisaged that other Mic-E-compatible devices will be allocated their
own type codes in future.

Note: When Kenwood radios receive the status, they can only display a small
number of text characters:

Kenwood TH-D7:
20 characters

Kenwood TM-D700: 28 characters
Note: The Kenwood TM-D700 radio uses the ' (apostrophe) instead of the ‘
(grave) APRS Data Type Identifier to represent current GPS data. A
suggested way of detecting this situation is to examine the first and 10th
characters of the Information field; if they are ' and ] respectively, then the
packet is almost certainly from a TM-D700.

\subsection{Maidenhead Locator in the Mic-E Status Text Field}

The Mic-E status text field can contain a Maidenhead locator.
If the locator is followed by a plain text comment, the first character of the
text must be a space. For example:

IO91SX/GV
V HelloV world
V HelloV world
>IO91SX/GV
]IO91SX/GV
V HelloV world

(from a Mic-E or PIC-E)
(from a Kenwood TH-D7)
(from a Kenwood TM-D700)

(/G is the grid locator symbol).

Altitude in the
Mic-E Status Text
Field

The Mic-E status text field can contain the station’s altitude. The altitude is
expressed in the form xxx}, where xxx is in meters relative to 10km below
mean sea level (the deepest ocean), to base 91.
For example, to compute the xxx characters for an altitude of 200 feet:
200 feet = 61 meters = 10061 meters relative to the datum
10061 / 912 = 1, remainder 1780
1780 / 91
= 19, remainder 51
Adding 33 to each of the highlighted values gives 34, 52 and 84 for the
ASCII codes of xxx.


Thus the 4-character altitude string is "4T}
Some examples:
"4T}
>"4T}
]"4T}

Mic-E Data in
Non-APRS
Networks

Some parts of the Mic-E AX.25 Information field may contain binary data
(i.e. non-printable ASCII characters). If such a packet is constrained to the
APRS network, this should not cause any difficulties.
If, however, the packet is to be forwarded via a network that does not reliably
preserve binary data (e.g. the Internet), then it is necessary to convert the data
to a format that will preserve it.
Further, if the packet subsequently re-emerges back onto the APRS network,
it will then be necessary to re-convert the data back to its original format.


