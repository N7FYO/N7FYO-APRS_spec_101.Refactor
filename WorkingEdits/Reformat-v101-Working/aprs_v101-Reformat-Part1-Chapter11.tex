\Chapter{Chapter 11: Object and Item Reports}

\section{Objects and Items}

Any APRS station can manually report the position of an APRS entity (e.g.
another station or a weather phenomenon). This is intended for situations
where the entity is not capable of reporting its own position.
APRS provides two types of report to support this:

\begin{itemize}

 \item Object Reports

 \item Item Reports

\end{itemize}


Object Reports specify an Object’s position, can have an optional timestamp,
and can include course/speed information or other Extended Data. Object
Reports are intended primarily for plotting the positions of moving objects
(e.g. spacecraft, storms, marathon runners without trackers).
Item Reports specify an Item’s position, but cannot have a timestamp. While
Item reports may also include course/speed or other Extended Data, they are
really intended for inanimate things that are occasionally posted on a map
(e.g. marathon checkpoints or first-aid posts). Otherwise they are handled in
the same way as Object Reports.

Objects are distinguished from each other by having different Object names.
Similarly, Items are distinguished from each other by having different Item
names.

Implementation Recommendation: When an APRS Object/Item is displayed
on the screen, the callsign of the station sending the report should be
associated with the Object/Item.

Replacing an
Object / Item

A fundamental precept of APRS is that any station may take over the
reporting responsibility for an APRS Object or Item, by simply transmitting a
new report with the same Object/Item name.

The replacement report may specify the existing location or a new location.
The original station will cease transmitting an Object/Item Report when it
sees an incoming report with the same name from another station.

Killing an
Object / Item

To kill an Object/Item, a station transmits a new Object/Item Report, with a
“kill” character following the Object/Item name.

Implementation Recommendation: When an Object/Item is killed it should be
removed from display on the screen. However, the data associated with the
Object/Item should be retained internally in case it is needed later.


Object Report
Format

An Object Report has a fixed 9-character Object name, which may consist of
any printable ASCII characters.
Object names are case-sensitive.

The ; is the APRS Data Type Identifier for an Object Report, and a * or _
separates the Object name from the rest of the report:

\begin{itemize} % no bullets?
  
\item * indicates a live Object.
\item _ indicates a killed Object.
\end{itemize}

The position may be in lat/long or compressed lat/long format, and the report
may also contain Extended Data.

An Object always has a timestamp.

The Comment field may contain any appropriate APRS data (see the
Comment Field section in Chapter 5: APRS Data in the AX.25 Information
Field).

Object Report Format — with Lat/Long position

;

Object
Name

*
or
_

Time
DHM /
HMS

Lat

Course/Speed

Sym
Table
ID

Long

Symbol
Code

Power/Height/Gain/Dir
Radio Range
DF Signal Strength

Comment
(max 36 chars with
Data Extension, or
43 without)

Area Object
Bytes:

1

9

1

7

8

1

9

1

Examples
;LEADERVVV*092345z4903.50N/07201.75W>088/036

;LEADERVVV_092345z4903.50N/07201.75W>088/036

7

0-36/43

A live Object. At 2345 hours zulu
on the 9th of the month, the
“Leader” was in the car at
49°3'30"N/72°1'45"W, heading 88
deg at 36 knots.
The same Object, now killed.

Object Report Format — with Compressed Lat/Long position

;

Bytes:

1

Object
Name

9

*
or
_
1

Time
DHM /
HMS

Compressed Position
Data

Comment

/YYYYXXXX$csT
7

13

43

Example
;LEADERVVV*092345z/5L!!<*e7>7P[



The “Leader” was in the car at 49°30'00"N/
72°45'00"W, heading 88 deg at 36.2 knots.



Item Report
Format

An Item Report has a variable-length Item name, 3–9 characters long. The
name may consist of any printable ASCII characters except ! or _.
Item names are case-sensitive.
The ) is the APRS Data Type Identifier for an Item Report, and a ! or _
separates the Item name from the rest of the report:
! indicates a live Item.
_ is the Item “kill” character.

The position may be in lat/long or compressed lat/long format. There is no
provision for a timestamp. The report may also contain Extended Data.
The Comment field may contain any appropriate APRS data (see the
Comment Field section in Chapter 5: APRS Data in the AX.25 Information
Field).
Item Report Format — with Lat/Long position
Item
Name

)

!
or
_

Lat

Sym
Table
ID

Course/Speed
Long

Symbol
Code

Power/Height/Gain/Dir
Radio Range

Comment
(max 36 chars with Data
Extension, or 43 without)

DF Signal Strength
Area Object

Bytes:

1

3-9

1

8

1

9

Examples
)AIDV#2!4903.50N/07201.75WA

1

7

0-36/43

First Aid Station #2 is at 49°3'30"N/72°1'45"W.
(/A is the symbol for Aid Station).

)G/WB4APR!53VV.VVN\002VV.VVWd

A rare DX station “somewhere in England”.
(\d is the symbol for DX Spot).

)AIDV#2_4903.50N/07201.75WA

The First Aid Station has closed down.

Item Report Format — with Compressed Lat/Long position

)

Bytes:

1

Item
Name

3-9

!
or
_

Compressed Position
Data

Comment

/YYYYXXXX$csT

1

Example
)MOBIL!\5L!!<*e79VsT

13

43

Mobil Gas Station is at 49°30'00"N/72°45'00"W.
(\9 is the symbol for Gas Station).


Area Objects

Using the \l symbol (i.e. the lower-case letter “L” symbol from the Alternate
Symbol Table) it is possible to define circle, line, ellipse, triangle and box
objects in all colors, either open or filled in, any size from 60 feet to 100
miles.
These Objects are useful for real-time events such as for a search-and-rescue,
or adding a special road or route for a special event.
The Object format is specified as a 7-character APRS Data Extension
Tyy/Cxx immediately following the l Symbol Code. For example:
;OBJECTVVV*ddmm.hhN\dddmm.hhW l Tyy/Cxx

where:
T is the type of object shape.
/C is the color of the object.
yy is the square root of the latitude offset in 1/100ths of a degree.
xx is the square root of the longitude offset in 1/100ths of a degree.

The object type and color codes are as follows:
T
0
1
2
3
4
5
6
7
8
9

Object Type
Open circle
Line (offset: down/right)
Open ellipse
Open triangle
Open box
Color-filled circle
Line (offset: down/left)
Color-filled ellipse
Color-filled triangle
Color-filled box

/C

Object Color

Intensity

/0
/1
/2
/3
/4
/5
/6
/7
/8
/9
10
11
12
13
14
15

Black

High

Blue

High

Green

High

Cyan

High

Red

High

Violet

High

Yellow

High

Gray

High

Black

Low

Blue

Low

Green

Low

Cyan

Low

Red

Low

Violet

Low

Yellow

Low

Gray

Low

The latitude/longitude position is the upper left corner of the object, and the
offsets are relative to this position — the yy offset is down from this position
and the xx offset is to the right of this position. (An exception is the special
case of a Type 6 line which is drawn down and to the left).



Here are some examples of Object Position Reports. The latitude and
longitude offsets are each one degree (i.e. 100/100ths of a degree), so
yy = xx = –100 = 10.
;SEARCHVVV*092345z4903.50N\07201.75W l 710/310

A high intensity cyan filled ellipse, yy=10, xx=10
;SEARCHVVV*092345z4903.50N\07201.75W l 8101310

A low intensity violet filled triangle, yy=10, xx=10
Further, with the line option (Type 1 and Type 6) it is possible to specify a
“corridor” either side of the central line. The width of the corridor (in miles)
either side of the line is specified in the comment text, enclosed by {}.
For example:
;FLIGHTPTH*4903.50N\07201.75W l 610/310{100}

A high intensity cyan line, with a 100-mile corridor
either side
Note: The color fill option should be used with care, since a color-filled
object will obscure information displayed underneath it.

Signpost
Objects/Items

Signpost Objects/Items (with the symbol \m) display as a yellow box with a
1–3-character overlay on them. The overlay is specified by enclosing the 1–3
characters in braces in the comment field. Thus a signpost with {55} would
appear as a sign with 55 on it.
For example:
)I91V3N!4903.50N\07201.75Wm{55}

This was originally designed for posting the speed of traffic past speed
measuring devices, but can be used for any purpose.
Implementation Recommendation: Signposts should not display any callsign
or name, and to avoid clutter should only be displayed at close range.

Obsolete Object
Format

Some stations transmit Object reports without the ; APRS Data Type
Identifier. This format is obsolete. Some software may still decode such data
as an Object, but it should now be interpreted as a Status Report.
