\chapter{Chapter 12: Weather Reports}

\section{Weather Report Types}

APRS is an ideal tool for reporting weather conditions via packet. APRS
supports serial data transmissions from the Peet Brothers, Ultimeter and
Davis home weather stations. It is even possible to mount an Ultimeter
remotely with only a TNC and radio to report and plot conditions. APRS is
also ideally suited for the Skywarn weather observer initiative.
APRS supports three types of Weather Report:

\section{Data Type Identifiers}
\begin{itemize}
\item Raw Weather Report
\item Positionless Weather Report
\item Complete Weather Report
\end{itemize}

The following APRS Data Type Identifiers are used in Weather Reports
containing raw data:

\begin{description}
\item \hl{!} Ultimeter 2000
\item \hl{\#} Peet Bros U-II
\item \hl{\$} Ultimeter 2000
\item \hl{*} Peet Bros U-II
\item \hl{\_} Positionless weather data
\end{description}


In addition, where the raw data has been post-processed (for example,
by the insertion of station location information), the four position
Data Type Identifiers \hl{!}, \hl{$=$}, \hl{/} and \hl{@} may be used
instead. In this case, the Weather Report is identified with the
weather symbol \hl{/\_} or \hl{\textbackslash \_} in the APRS Data.


\section {Raw Weather Reports}

Raw weather data from a stand-alone weather station is contained in the
Information Field of an APRS AX.25 frame:


  % top caption   & Raw Weather Report Format & & \\
\begin{tabular}{|l|c|r|}
  \hline
  \multicolumn{3}{|c|}{Raw Weather Report Format} \\
  \hline
  \hline
  & ! or &  \\
  & \# or  & Raw Weather Data \\
  & \$ or  & \\
  & *  & \\
  \hline 
  Bytes: & 1 & n   \\
  \hline
\end{tabular}



\underline{Examples}
\begin{description}
\item [Ultimeter 2000: ] !!006B005803500000----03E9--------002105140000005D (Ultimeter 2000)
\item [Peet Bros U-II: ] \#50B7500820082   
\item [Ultimeter 2000: ] \$ULTW0031003702CE0069----000086A00001----011901CC00000005   
\item [Peet Bros U-II: ] *7007600000000  
\end{description}


\section{Positionless Weather Reports}

Generic raw weather data from a stand-alone weather station is contained in
the Information Field of an APRS AX.25 frame:

% begin chart 

    %  Top Caption: Positionless Weather Report Format   
\begin{tabular}{|l|c|c|c|c|r|}
  \hline
  \multicolumn{6}{c}{Positionless Weather Report Format} \\
  \hline
  & & & & &  \\ 
  & \_ & Time (MDHM) & Positionless Weather Data & APRS Software (S) & WX Unit (uuuu) \\
  & & & & &  \\ 
  \hline
  & & & & &  \\ 
  Bytes: & 1 & 8 & n & 1 & 2-4 \\
  & & & & &  \\ 
  \hline
\end{tabular}

\underline{Example}

\begin{description}
\item [Report derived from Radio Shack WX station data:]
\item \_10090556c220s004g005t077r000p000P000h50b09900wRSW 
\end{description}


\subsection {APRS Software Type}

A Weather Report may contain a single-character code S for the type of
APRS software that is running at the weather station:

% change to a description list

\begin{description}
\item \hl{d} = APRSdos
\item \hl{M} = MacAPRS
\item \hl{P} = pocketAPRS
\item \hl{S} = APRS+SA
\item \hl{W} = WinAPRS
\item \hl{X} = X-APRS (Linux)
\end{description}

\subsubsection {Weather Unit Type}

A Weather Report may contain a 2–4 character code uuuu for the type of
weather station unit. The following codes have been allocated:

\begin{description}
\item \hl{Dvs} = Davis
\item \hl{HKT} = Heathkit
\item \hl{PIC} = PIC device
\item \hl{RSW} = Radio Shack
\item \hl{U-II} = Original Ultimeter U-II (auto mode)
\item \hl{U2R} = Original Ultimeter U-II (remote mode)
\item \hl{U2k} = Ultimeter 500/2000
\item \hl{U2kr} = Remote Ultimeter logger
\item \hl{U5} = Ultimeter 500
\item \hl{Upkm} = Remote Ultimeter packet mode
\end{description}

  Users may specify any other 2–4 character code for devices not in this list.

% Should a new list like https://github.com/aprsorg/aprs-deviceid be created for weather devices?


\subsection {Positionless Weather Data}

The format of weather data within a Positionless Weather Report differs
according to the type of weather station unit, but generically consists of some
or all of the following elements:

% Begin chart

\begin{tabular}{r|c|c|c|c|c|c|c|c|c|}
  \hline
  \multicolumn{10}{c}{Positionless Weather Data} \\
  \hline
  & Wind  & Wind & Gust & Temp & Rain &  & Rain  & Humidity & Barometric  \\
  & Direction & Speed & & (F) & Last Hr & Last 24 Hrs & Since Midnight & & Pressure \\
  & & & & & & & &  & \\
  & cccc & ssss & gggg & tttt & rrrr & pppp & PPPP & hhh & bbbbbb \\
  \hline 
  Bytes & 4 & 4 & 4 & 4 & 4 & 4 & 4 & 3 & 5 \\
  \hline 
  
\end{tabular}

Where:

\begin{itemize}
\item c = wind direction (in degrees).
\item s = sustained one-minute wind speed (in mph).
\item g = gust (peak wind speed in mph in the last 5 minutes).
\item t = temperature (in degrees Fahrenheit). Temperatures below zero are expressed as -01 to -99.
\item r = rainfall (in hundredths of an inch) in the last hour.
\item p = rainfall (in hundredths of an inch) in the last 24 hours.
\item P = rainfall (in hundredths of an inch) since midnight.
\item h = humidity (in \%. 00 = 100\%).
\item b = barometric pressure (in tenths of millibars/tenths of hPascal).
\end{itemize}



Other parameters that are available on some weather station units include:
\begin{itemize}
\item L = luminosity (in watts per square meter) 999 and below.

\item l (lower-case letter “L”) = luminosity (in watts per square meter) 1000 and above.
\item NB: (L is inserted in place of one of the rain values).
\item s= snowfall (in inches) in the last 24 hours.
\item \#= raw rain counter
\end{itemize}

Note: The weather report must include at least the MDHM date/timestamp,
wind direction, wind speed, gust and temperature, but the remaining
parameters may be in a different order (or may not even exist).
Note: Where an item of weather data is unknown or irrelevant, its value may
be expressed as a series of dots or spaces. For example, if there is no wind
speed/direction/gust sensor, the wind values could be expressed as:
\begin{itemize}
\item c...s...g...
\item  or: c\textvisiblespace\textvisiblespace\textvisiblespace s\textvisiblespace\textvisiblespace\textvisiblespace g \textvisiblespace\textvisiblespace\textvisiblespace
\end{itemize}

For example, Jim’s rain gauge may produce a report like this:\\
\_10090556c...s...g...t...P012Jim

(The date/timestamp, wind direction/speed/gust and temperature parameters
must be included, even though they are not meaningful).


\subsection{Location of a Raw and Positionless Weather Stations}

APRS cannot display weather data on a map until it knows the location of the
sending station. In the case of a station transmitting Raw or Positionless
Weather Reports, the station has to occasionally send an additional packet
containing its position (using any of the legal lat/long and compressed
lat/long position formats described earlier).

\subsection {Symbols with Raw and Positionless Weather Stations}

Because Raw and Positionless Weather Reports do not contain a display
symbol in the AX.25 Information field, it is possible to specify the symbol in
a generic APRS destination address (e.g. GPSHW or GPSE63) instead.
Alternatively, if the weather station is on a balloon, the SSID –11 may be
used in the source address (e.g. N0QBF-11).

See Chapter 20: APRS Symbols for more detail on the usage of symbols.

\section {Complete Weather Reports with Timestamp and Position}

An APRS Complete Weather Report can contain a timestamp and location
information, using any of the legal lat/long and compressed lat/long position
formats described earlier. An APRS Object may also have weather
information associated with it.

Examples of report formats are shown below. Note that the Symbol Code in
every case is the \_ (underscore). Also, the 7-byte Wind Direction and Wind
Speed Data Extension replace the cccc and ssss fields of a Positionless
Weather Report.




\begin{tabular}{|l||c|c|c|c|c|c|c|c|c|}
  \hline 
    \multicolumn{10}{|c|}{ } \\
    \multicolumn{10}{|c|}{Complete Weather Report Format — with Lat/Long position, no Timestamp} \\
        \multicolumn{10}{|c|}{ }\\
  \hline
  & & & & & & & & & \\
  & !  & Lat & Sym & Long & Symbol & Wind & Weather & APRS & WX \\
  & or & & Table & & Code & Direction & Data & Software & Unit \\
  & = & & ID & & & Speed  & & & \\
  &  & & & & \_ & & & S & uuuu \\
  & & & & & & & & & \\
  \hline 
  Bytes & & & & & & & & & \\
  & & & & & & & & & \\
  & 1 & 8 & 1 & 9 & 1 & 7 & n & 1 & 2-4  \\
  \hline

\end{tabular}


\underline{Examples}
\begin{description}
\item !4903.50N/07201.75W\_220/004g005t077r000p000P000h50b09900wRSW
\item !4903.50N/07201.75W\_220/004g005t077r000p000P000h50b.....wRSW
\end{description}



\begin{tabular}{|l||c|c|c|c|c|c|c|c|c|c|}
  \hline 
    \multicolumn{11}{|c|}{ } \\
    \multicolumn{11}{|c|}{Complete Weather Report Format — with Lat/Long position and Timestamp} \\
        \multicolumn{11}{|c|}{ }\\
  \hline
  & & & & & & & & & & \\
  & / & Time  & Lat & Sym & Long & Symbol & Wind & Weather & APRS & WX \\
  & or & DHM/ & & Table & & Code & Direction & Data & Software & Unit \\
  & @ & HMS & & ID & & & Speed  & & & \\
  &  & & & & & \_ & & & S & uuuu \\
  & & & & & & & & & & \\
  \hline 
  Bytes & & & & & & & & & & \\
  & & & & & & & & & & \\
  & 1 & 7 & 8 & 1 & 9 & 1 & 7 & n & 1 & 2-4  \\
  \hline

\end{tabular}

\underline{Example}
\begin{description}
\item @092345z4903.50N/07201.75W\_220/004g005t-07r000p000P000h50b09900wRSW
\end{description}

\vskip10ex

\begin{tabular}{|l||c|c|c|c|c|c|c|c|c|c|}
  \hline
  \multicolumn{11}{|c|}{ }\\
\multicolumn{11}{|c|}{Complete Weather Report Format — with Compressed Lat/Long position, no Timestamp} \\
\multicolumn{11}{|c|}{} \\
\multicolumn{11}{|c|}{ }\\
\hline
  & & & & & & & & & & \\
  & ! & Sym   & Comp &  Comp  &  Symbol & Comp & Comp & Weather & APRS & WX \\
  & or & Table & Lat & Long &  Code & Wind Dir  & Type & Data & Software & Unit \\
  & = &  ID & & & & + Speed  & & & & \\
  & & & YYYY & XXXX & \_ & & T & & S & uuuu \\
  & & & & & & & & & & \\
  \hline 
  Bytes & & & & & & & & & & \\
  & & & & & & & & & & \\
  & 1 & 1 & 4 & 4 & 1 & 2 & 1 & n & 1 & 2-4  \\
  & & & & & & & & & & \\
  \hline

\end{tabular}

\underline{Example}
\begin {description}
\item =/5L!!<*e7>\_7P[g005t077r000p000P000h50b09900wRSW
\end{description}

\vskip10ex
  


\begin{tabular}{|l||c|c|c|c|c|c|c|c|c|c|c|}
  \hline
  \multicolumn{12}{|c|}{ }\\
\multicolumn{12}{|c|}{Complete Weather Report Format — with Compressed Lat/Long position, with Timestamp} \\
\multicolumn{12}{|c|}{} \\
\multicolumn{12}{|c|}{ }\\
\hline
  & & & & & & & & & & & \\
  & / & Time & Sym   & Comp &  Comp  &  Symbol & Comp & Comp & Weather & APRS & WX \\
  & or & & Table & Lat & Long &  Code & Wind Dir  & Type & Data & Software & Unit \\
  & @ & DHM & ID & & & & + Speed  & & & & \\
  & & HMS & & YYYY & XXXX & \_ & & T & & S & uuuu \\
  & & & & & & & & & & & \\
  \hline 
  Bytes & & & & & & & & & & & \\
  \hline
  & & & & & & & & & & & \\
  & 1 & 7 & 1 & 4 & 4 & 1 & 2 & 1 & n & 1 & 2-4  \\
  & & & & & & & & & & & \\
  \hline

\end{tabular}
\vskip 5ex

\underline{Example}
\begin{description}
\item @092345z/5L!!<*e7\_7P[g005t077r000p000P000h50b09900wRSW
\end{description}

\vskip 10ex
  


\begin{tabular}{|l||c|c|c|c|c|c|c|c|c|c|c|c|}
  \hline
  \multicolumn{13}{|c|}{ }\\
\multicolumn{13}{|c|}{Complete Weather Report Format — with Object and Lat/Long position} \\
\multicolumn{13}{|c|}{} \\
\multicolumn{13}{|c|}{ }\\
\hline
  & & & & & & & & & & & & \\
  & & Object & & Time & Lat & Sym & Long & Symbol & Wind & Weather & APRS & WX \\
  & & Name & & DHM/ & & Table & & & Directn/ & Data & Software & Unit \\
  & & & & HMS & & ID & & & Speed & & & \\
  & & & & & & & & & &  & S & uuuu \\
  & ; & & * & & & & & \_ & & & & \\
  & & & & & & & & & & & & \\
  \hline 
  Bytes & & & & & & & & & & & & \\
  \hline
  & & & & & & & & & & & & \\
  & 1 & 9 & 1 & 7 & 8 & 1 & 9 & 1 & 7 & n & 1 & 2-4  \\
  & & & & & & & & & & & & \\
  \hline

\end{tabular}

\vskip 5ex

\underline{Examples}

\begin{description}
\item ;BRENDAVVV*4903.50N/07201.75W\_220/004g005t077r000p000P000h50b09900wRSW
\item ;BRENDAVVV*092345z4903.50N/07201.75W\_220/004g005b0990
\end{description}

\section{Storm Data}

APRS reports can contain data relating to tropical storms, hurricanes and
tropical depressions. The format of the data is as follows:

% begin chart

\begin{tabular}{|l|c|c|c|c|c|c|c|c|c|c|}
  \hline
  \multicolumn{11}{|c|}{ } \\
  \multicolumn{11}{|c|}{Storm Data} \\
  \hline
  & & & & & & & & & & \\
  & & & & Storm & Sustained  & Peak & Central  & Radius & Radius & Radius \\
  & & & & Type & Wind & Wind & Pressure  & Hurricane & Tropical & Whole \\
  & Direction & / & Speed & & Speed & Gusts & & Winds & Storm & Gale  \\
  & & & & & & & & & & \\
  & & & & & & & & & & \\
  & & & & /ST & /www & \textasciicircum GGG & /ppp & >RRR & \&rrr & \%ggg  \\ 
  \hline
  & & & & & & & & & & \\ 
  Bytes & 3 & 1 & 3 & 3 & 4 & 4 & 5 & 4 & 4 & 4 \\
  & & & & & & & & & & \\
  \hline
\end{tabular}

\vskip 5ex
Where ST is one of:

\begin{description}
\item = TS (Tropical Storm)
\item = HC (Hurricane)
\item = TD (Tropical Depression)
\end{description}

and
\begin{description}
\item WWW = sustained wind speed (in knots).
\item GGG = gust (peak wind speed in knots).
\item pppp = central pressure (in millibars/hPascal)
\item RRR = radius of hurricane winds (in nautical miles).
\item rrr = radius of tropical storm winds (in nautical miles).
\item ggg = radius of “whole gale” (i.e. 50 knot) winds (in nautical
  miles). Optional.
\end {description}


Storm data will usually be included in an Object Report, but may also be
included in a Position Report or an Item Report.


The display symbol will be either:

\begin{description}
\item \hl{\textbackslash@} Hurricane/Tropical Storm (current position)
\item \hl{/@} Hurricane (predicted future position)
\end{description}

For example, the progress of Hurricane Brenda could be expressed in Object
Reports like these:

\begin{verbatim}
;BRENDAVVV*092345z4903.50N\07202.75W@088/036/HC/150^200/0980>090&030%040
;BRENDAVVV*100045z4905.50N/07201.75W@101/047/HC/104^123/0980>065&020%040
\end{verbatim}

% Editors note: seems to be a completely hypothetical hurricane in an
% inland portion of Quebec Province

\section {National Weather Service Bulletins}

APRS supports the dissemination of National Weather Service bulletins. See
Chapter 14: Messages, Bulletins and Announcements.



