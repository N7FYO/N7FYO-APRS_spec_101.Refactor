\chapter{Chapter 12: Weather Reports}

\section{Weather Report Types}

APRS is an ideal tool for reporting weather conditions via packet. APRS
supports serial data transmissions from the Peet Brothers, Ultimeter and
Davis home weather stations. It is even possible to mount an Ultimeter
remotely with only a TNC and radio to report and plot conditions. APRS is
also ideally suited for the Skywarn weather observer initiative.
APRS supports three types of Weather Report:

Data Type
Identifiers

•

Raw Weather Report

•

Positionless Weather Report

•

Complete Weather Report

The following APRS Data Type Identifiers are used in Weather Reports
containing raw data:
!
#
$
*
_

Ultimeter 2000
Peet Bros U-II
Ultimeter 2000
Peet Bros U-II
Positionless weather data

In addition, where the raw data has been post-processed (for example, by the
insertion of station location information), the four position Data Type
Identifiers !, =, / and @ may be used instead. In this case, the Weather
Report is identified with the weather symbol /_ or \_ in the APRS Data.

Raw Weather
Reports

Raw weather data from a stand-alone weather station is contained in the
Information Field of an APRS AX.25 frame:

Raw Weather Report Format
! or
# or
$ or
*
Bytes:

1

Raw Weather
Data

n

Examples
\begin{verbatim}
!!006B005803500000----03E9--------002105140000005D
#50B7500820082
$ULTW0031003702CE0069----000086A00001----011901CC00000005
*7007600000000
\end{verbatim}


Ultimeter 2000
Peet Bros U-II
Ultimeter 2000
Peet Bros U-II


\subsection{Positionless Weather Reports}

Generic raw weather data from a stand-alone weather station is contained in
the Information Field of an APRS AX.25 frame:

Positionless Weather Report Format

Bytes:

_

Time
MDHM

1

8

Positionless Weather
Data
n

APRS
Software

WX
Unit

S

uuuu

1

2-4

Example
_10090556c220s004g005t077r000p000P000h50b09900wRSW
report derived from Radio Shack WX station data.

APRS Software Type

A Weather Report may contain a single-character code S for the type of
APRS software that is running at the weather station:
\begin{verbatim}
d = APRSdos
M = MacAPRS
P = pocketAPRS
S = APRS+SA
W = WinAPRS
X = X-APRS (Linux)
\end{verbatim}

Weather Unit
Type

A Weather Report may contain a 2–4 character code uuuu for the type of
weather station unit. The following codes have been allocated:
Dvs

= Davis

HKT

= Heathkit

PIC

= PIC device

RSW

= Radio Shack

U-II = Original Ultimeter U-II (auto mode)
U2R

= Original Ultimeter U-II (remote mode)

U2k

= Ultimeter 500/2000

U2kr = Remote Ultimeter logger
U5

= Ultimeter 500

Upkm = Remote Ultimeter packet mode
Users may specify any other 2–4 character code for devices not in this list.


Positionless Weather Data

The format of weather data within a Positionless Weather Report differs
according to the type of weather station unit, but generically consists of some
or all of the following elements:

Positionless Weather Data

Bytes:

Wind
Direction

Wind
Speed

Gust

Temp

Rain
Last Hr

Rain
Last 24 Hrs

Rain
Since Midnight

Humidity

Barometric
Pressure

cccc

ssss

gggg

tttt

rrrr

pppp

PPPP

hhh

bbbbbb

4

4

4

4

4

4

4

3

5

where: c =
s =
g =
t =
r =
p =
P =
h =
b =

wind direction (in degrees).
sustained one-minute wind speed (in mph).
gust (peak wind speed in mph in the last 5 minutes).
temperature (in degrees Fahrenheit). Temperatures below
zero are expressed as -01 to -99.
rainfall (in hundredths of an inch) in the last hour.
rainfall (in hundredths of an inch) in the last 24 hours.
rainfall (in hundredths of an inch) since midnight.
humidity (in %. 00 = 100%).
barometric pressure (in tenths of millibars/tenths of hPascal).

Other parameters that are available on some weather station units include:
L =

luminosity (in watts per square meter) 999 and below.

l

(lower-case letter “L”) = luminosity (in watts per square meter)
1000 and above.
(L is inserted in place of one of the rain values).
s=
snowfall (in inches) in the last 24 hours.
#=
raw rain counter
Note: The weather report must include at least the MDHM date/timestamp,
wind direction, wind speed, gust and temperature, but the remaining
parameters may be in a different order (or may not even exist).
Note: Where an item of weather data is unknown or irrelevant, its value may
be expressed as a series of dots or spaces. For example, if there is no wind
speed/direction/gust sensor, the wind values could be expressed as:
c...s...g...

or cVVV
VVVsVVV
VVVgVVV
VVV

For example, Jim’s rain gauge may produce a report like this:
_10090556c...s...g...t...P012Jim

(The date/timestamp, wind direction/speed/gust and temperature parameters
must be included, even though they are not meaningful).


\section{Location of a Raw and Positionless Weather Stations}

APRS cannot display weather data on a map until it knows the location of the
sending station. In the case of a station transmitting Raw or Positionless
Weather Reports, the station has to occasionally send an additional packet
containing its position (using any of the legal lat/long and compressed
lat/long position formats described earlier).

Symbols with Raw
and Positionless
Weather Stations

Because Raw and Positionless Weather Reports do not contain a display
symbol in the AX.25 Information field, it is possible to specify the symbol in
a generic APRS destination address (e.g. GPSHW or GPSE63) instead.
Alternatively, if the weather station is on a balloon, the SSID –11 may be
used in the source address (e.g. N0QBF-11).
See Chapter 20: APRS Symbols for more detail on the usage of symbols.

Complete Weather
Reports with
Timestamp and
Position

An APRS Complete Weather Report can contain a timestamp and location
information, using any of the legal lat/long and compressed lat/long position
formats described earlier. An APRS Object may also have weather
information associated with it.
Examples of report formats are shown below. Note that the Symbol Code in
every case is the _ (underscore). Also, the 7-byte Wind Direction and Wind
Speed Data Extension replace the cccc and ssss fields of a Positionless
Weather Report.

Complete Weather Report Format — with Lat/Long position, no Timestamp

! or
=

Lat

Sym
Table
ID

Long

Symbol
Code

Wind
Directn/
Speed

Weather
Data

_
Bytes:

1

8

1

9

1

7

n

APRS
Software

WX
Unit

S

uuuu

1

2-4

Examples
!4903.50N/07201.75W_220/004g005t077r000p000P000h50b09900wRSW
!4903.50N/07201.75W_220/004g005t077r000p000P000h50b.....wRSW



Complete Weather Report Format — with Lat/Long position and Timestamp

/ or
@

Time
DHM /
HMS

Sym
Table
ID

Lat

Wind
Directn/
Speed

Symbol
Code

Long

Weather
Data

APRS
Software

WX
Unit

S

uuuu

1

2-4

_
Bytes:

1

7

8

1

9

1

7

n

Example
@092345z4903.50N/07201.75W_220/004g005t-07r000p000P000h50b09900wRSW
Complete Weather Report Format — with Compressed Lat/Long position, no Timestamp

! or
=
Bytes:

1

Sym
Table
ID

Comp
Type

_

Comp
Wind
Directn/
Speed

1

2

1

Comp
Lat

Comp
Long

Symbol
Code

YYYY

XXXX

4

4

1

Weather
Data

T
n

APRS
Software

WX
Unit

S

uuuu

1

2-4

Example
=/5L!!<*e7> _7P[g005t077r000p000P000h50b09900wRSW
Complete Weather Report Format — with Compressed Lat/Long position, with Timestamp

/ or
@
Bytes:

1

Time
DHM /
HMS

Sym
Table
ID

7

Comp
Lat

Comp
Long

Symbol
Code

YYYY

XXXX

4

4

1

Comp
Type

_

Comp
Wind
Directn/
Speed

1

2

1

Weather
Data

T
n

APRS
Software

WX
Unit

S

uuuu

1

2-4

APRS
Software

WX
Unit

S

uuuu

1

2-4

Example
@092345z/5L!!<*e7 _7P[g005t077r000p000P000h50b09900wRSW
Complete Weather Report Format — with Object and Lat/Long position
Object
Name

;

*

Time
DHM /
HMS

Lat

Sym
Table
ID

Long

Symbol
Code

Wind
Directn/
Speed

Weather
Data

_
Bytes:

1

9

1

7

8

1

9

1

7

n

Examples
;BRENDAVVV*4903.50N/07201.75W_220/004g005t077r000p000P000h50b09900wRSW
;BRENDAVVV*092345z4903.50N/07201.75W_220/004g005b0990



\paragraph{Storm Data}

APRS reports can contain data relating to tropical storms, hurricanes and
tropical depressions. The format of the data is as follows:

Storm Data

Direction

Bytes:

3

/

Storm
Type

Sustained
Wind
Speed

Peak
Wind
Gusts

Central
Pressure

Radius
Hurricane
Winds

/ST

/www

^GGG

/pppp

>RRR

Speed

1

3

3

where: ST

4

=

4

5

4

Radius
Tropical
Storm
Winds

Radius
Whole
Gale

&rrr

%ggg

4

4

TS (Tropical Storm)
HC (Hurricane)
TD (Tropical Depression).

sustained wind speed (in knots).
GGG = gust (peak wind speed in knots).
pppp = central pressure (in millibars/hPascal)
RRR = radius of hurricane winds (in nautical miles).
rrr = radius of tropical storm winds (in nautical miles).
ggg = radius of “whole gale” (i.e. 50 knot) winds (in
nautical miles). Optional.
Storm data will usually be included in an Object Report, but may also be
included in a Position Report or an Item Report.
www

=

The display symbol will be either:
\@
/@

Hurricane/Tropical Storm (current position)
Hurricane (predicted future position)

For example, the progress of Hurricane Brenda could be expressed in Object
Reports like these:
;BRENDAVVV*092345z4903.50N\07202.75W@088/036/HC/150^200/0980>090&030%040
;BRENDAVVV*100045z4905.50N/07201.75W@101/047/HC/104^123/0980>065&020%040

National Weather
Service Bulletins

APRS supports the dissemination of National Weather Service bulletins. See
Chapter 14: Messages, Bulletins and Announcements.



