\chapter{Chapter 13: Telemetry Data}


The AX.25 Information field can contain telemetry data. The APRS Data
Type Identifier is T.

\section{Telemetry Report Format}

The report Sequence Number is a 3-character value — typically a 3-digit
number, or the three letters MIC. In the case of MIC, there may or may not be
a comma preceding the first analog data value.

There are five 8-bit unsigned analog data values (expressed as 3-digit
decimal numbers in the range 000–255), followed by a single 8-bit digital
data value (expressed as 8 bytes, each containing 1 or 0).

The Kantronics KPC-3+ TNC and APRS Micro Interface Module (MIM) use
this format.

Telemetry Report Format

Bytes:

Sequence
No

Analog
Value 1

Analog
Value 2

Analog
Value 3

Analog
Value 4

Analog
Value 5

Digital
Value

T

#xxx,

aaa,

aaa,

aaa,

aaa,

aaa,

bbbbbbbb

1

5

4

4

4

4

4

8

Comment

n

Examples
T#005,199,000,255,073,123,01101001
T#MIC199,000,255,073,123,01101001
On-Air Definition of
Telemetry
Parameters

In principle, received telemetry data may be interpreted in any appropriate
way. In practice, however, an APRS user can define the telemetry parameters
(such as quadratic coefficients for the analog values, or the meaning of the
binary data) at any time, and then send these definitions as APRS messages.
Other stations receiving these messages will then know how to interpret the
data.

This is achieved by sending four messages:
•

A Parameter Name message.

•

A Unit/Label message.

•

An Equation Coefficients message.

•

A Bit Sense/Project Name message.

The messages addressee is the callsign of the station transmitting the
telemetry data. For example, if N0QBF launches a balloon with the callsign
N0QBF-11, then the four messages are addressed to N0QBF-11.

See Chapter 14: Messages, Bulletins and Announcements for full details of
message formats.

The Parameter Name message contains the names (N) associated with the
five analog channels and the 8 digital channels. Its format is as follows:

Parameter Name
Message

Telemetry Parameter Name Message Data
Note the different byte counts, which include commas where shown. The list may stop at any field.

Bytes:

PARM.

A1
N…

A2
,N…

A3
,N…

A4
,N…

A5
,N…

B1
,N…

B2
,N…

B3
,N…

B4
,N…

B5
,N…

B6
,N…

B7
,N…

B8
,N…

5

1-7

1-7

1-6

1-6

1-5

1-6

1-5

1-4

1-4

1-4

1-3

1-3

1-3

Example
:N0QBF-11V:PARM.Battery,Btemp,ATemp,Pres,Alt,Camra,Chut,Sun,10m,ATV

Note: The field widths are not all the same (this is a legacy arising from
earlier limitations in display screen width). Note also that the byte counts
include the comma separators where shown.
The list can terminate after any field.

The Unit/Label message specifies the units (U) for the analog values, and the
labels (L) associated with the digital channels:

Unit/Label Message

Telemetry Unit/Label Message Data
Note the different byte counts, which include commas where shown. The list may stop at any field.

Bytes:

UNIT.

A1
U…

A2
,U…

A3
,U…

A4
,U…

A5
,U…

B1
,L…

B2
,L…

B3
,L…

B4
,L…

B5
,L…

B6
,L…

B7
,L…

B8
,L…

5

1-7

1-7

1-6

1-6

1-5

1-6

1-5

1-4

1-4

1-4

1-3

1-3

1-3

Example
:N0QBF-11V:UNIT.v/100,deg.F,deg.F,Mbar,Kft,Click,OPEN,on,on,hi

Note: Again, the field widths are not all the same, and the byte counts
include the comma separators where shown.
The list can terminate after any field.


The Equation Coefficients message contains three coefficients (a, b and c)
for each of the five analog channels.

Equation
Coefficients
Message

Telemetry Equation Coefficients Message Data
2
The list may stop at any field. Value = a x v + b x v + c
A1

Bytes:

A2

A3

A4

A5

EQNS.

a

,b

,c

,a

,b

,c

,a

,b

,c

,a

,b

,c

,a

,b

,c

5

n

n

n

n

n

n

n

n

n

n

n

n

n

n

n

Example
:N0QBF-11V:EQNS.0,5.2,0,0,.53,-32,3,4.39,49,-32,3,18,1,2,3

To obtain the final value of an analog channel, these coefficients are
substituted into the equation:

$ a x v2 + b x v + c $ 

where v is the raw received analog value.
For example, analog channel A1 in the above beacon examples relates to the
battery voltage, expressed in hundredths of volts, and a = 0, b = 5.2, c = 0. If
the raw received value v is 199, then the voltage is calculated as:
voltage = 0 x 1992 + 5.2 x 199 + 0
= 1034.8 hundredths of a volt
= 10.348 volts
The Bit Sense/Project Name message contains two types of information:

Bit Sense/
Project Name
Message

•

An 8-bit pattern of ones and zeros, specifying the sense of each digital
channel that matches the corresponding label.

•

The name of the project associated with the telemetry station.

Telemetry Bit Sense/Project Name Message Data

Bytes:

BITS.

B1
x

B2
x

B3
x

B4
x

B5
x

B6
x

B7
x

B8
x

Project Title

5

1

1

1

1

1

1

1

1

0-23

Example
:N0QBF-11V:BITS.10110000,N0QBF’s Big Balloon

Thus in the above message examples, if digital channel B1 is 1, this indicates
the camera has clicked. If channel B2 is 0, the parachute has opened, and so
on.
