\chapter{Chapter 14: Messages, Bulletins and Announcements}


APRS messages, bulletins and announcements are packets containing free
format text strings, and are intended to convey human-readable information.
A message is intended for reception by a single specified recipient, and an
acknowledgement is usually expected. Bulletins and announcements are
intended for reception by multiple recipients, and are not acknowledged.

An APRS message is a text string with a specified addressee. The addressee
is a fixed 9-character field (padded with spaces if necessary) following the :
Data Type Identifier. The addressee field is followed by another :, then the
text of the message.

\section{Messages}

The message text may be up to 67 characters long, and may contain any
printable ASCII characters except |, ~ or {.
A message may also have an optional message identifier, which is appended
to the message text. The message identifier consists of the character {
followed by a message number (up to 5 alphanumeric characters, no spaces)
to identify the message.

Messages without a message identifier are not to be acknowledged.
Messages with a message identifier are intended to be acknowledged by the
addressee. The sending station will repeatedly send the message until it
receives an acknowledgement, or it is canceled, or it times out.

Message Format
Message ID
Addressee

:
Bytes:

1

:
9

1

Examples
:WU2ZVVVVV:Testing

:WU2ZVVVVV:Testing{003

Message Text
(max 67 chars)
0-67

Message No

{

xxxxx

1

1-5

A message for WU2Z, containing the text “Testing”,
no acknowledgement expected.
(Note the filler spaces in the 9-character addressee field).
The same message, Message No=003, acknowledgement expected.

:EMAILVVVV:msproul@ap.org Test email

APRS Protocol Reference — APRS Protocol Version 1.0

An e-mail message (Note: This is an example
of how such a message could be constructed.
APRS itself does not support e-mail delivery)



Message
Acknowledgement

A message acknowledgement is similar to a message, except that the message
text field contains just the letters ack, and this is followed by the Message
Number being acknowledged.
Message Acknowledgement Format
Message
No

Addressee

:
Bytes:

1

9

:

ack

xxxxx

1

3

1–5

Example
:KB2ICI-14:ack003

Message Rejection

If a station is unable to accept a message, it can send a rej message instead
of an ack message:
Message Rejection Format
Message
No

Addressee

:
Bytes:

1

9

:

rej

xxxxx

1

3

1–5

Example
:KB2ICI-14:rej003
Multiple
Acknowledgements

If a station receives a particular message more than once, it will respond with
an acknowledgement for each instance of the message.
If a station receives a message over a long path, it may respond with a
reasonable number of multiple copies of the acknowledgement, to improve
the chances of the originating station receiving at least one of the copies.
In either of these two situations, multiple message acknowledgements should
be separated by at least 30 seconds (this is because some network
components such as digipeaters will suppress duplicated messages within a
30-second period).

Message Groups

An APRS receiving station can specify special Message Groups, containing
lists of callsigns that the station will read messages from (in addition to
messages addressed to itself). Such Message Groups are defined internally
by the user at the receiving station, and are used to filter received message
traffic.



The receiving station will read all messages with the Addressee field set to
ALL, QST or CQ.
The receiving station will only acknowledge messages addressed to itself,
and not any messages received which were addressed to any group callsign.
Note: The receiving station will acknowledge all messages addressed to
itself, even if it is operating in an Alternate Net (see Chapter 4: APRS Data
in the AX.25 Destination and Source Address Fields).

General bulletins are messages where the addressee consists of the letters

General Bulletins

BLN followed by a single-digit bulletin identifier, followed by 5 filler spaces.

General bulletins are generally transmitted a few times an hour for a few
hours, and typically contain time sensitive information (such as weather
status).
Bulletin text may be up to 67 characters long, and may contain any printable
ASCII characters except | or ~.
General Bulletin Format
Bulletin
ID
Bytes:

:

BLN

n

VVVVV

:

Bulletin Text
(max 67 characters)

1

3

1

5

1

0-67

Example
:BLN3VVVVV:Snow expected in Tampa RSN

Announcements are similar to general bulletins, except that the letters BLN
are followed by a single upper-case letter announcement identifier.
Announcements are transmitted much less frequently than bulletins (but
perhaps for several days), and although possibly timely in nature they are
usually not time critical.

Announcements

Announcements are typically made for situations leading up to an event, in
contrast to bulletins which are typically used within the event.
Users should be alerted on arrival of a new bulletin or announcement.
Announcement Format
Announcement
Identifier
Bytes:

:

BLN

x

VVVVV

:

Announcement Text
(max 67 characters)

1

3

1

5

1

0-67

Example
:BLNQVVVVV:Mt St Helen digi will be QRT this weekend


Bulletins may be sent to bulletin groups. A bulletin group address consists of
the letters BLN, followed by a single-digit group bulletin identifier, followed
in turn by the name of the group (up to 5 characters long, with filler spaces to
pad the name to 5 characters).

Group Bulletins

Group Bulletin Format
Group Bulletin
ID
Bytes:

:

BLN

n

Group
Name

:

Group Bulletin Text
(max 67 characters)

1

3

1

5

1

0-67

Example
:BLN4WXVVV:Stand by your snowplows

Group bulletin number 4 to the WX group.
(Note the filler spaces in the group name).

A receiving station can specify a list of bulletin groups of interest. The list is
defined internally by the user at the receiving station. If a group is selected
from the list, the station will only copy bulletins for that group, plus any
general bulletins. If the list is empty, all bulletins are received and generate
alerts.

Standard APRS message formats can be used for a variety of other
applications. For example, in the United States, special formatted messages
addressed to the generic callsign NWS-xxxxx are used to highlight map areas
involved in weather warnings, using the following format:

National Weather
Service Bulletins

National Weather Service Bulletin Format

Bytes:

:

NWS-xxxxx

:

NWS Bulletin Text

1

9

1

n

Example
:NWS-WARNV:092010z,THUNDER_STORM,AR_ASHLEY,{S9JbA
(Note: The “message identifier” {S9JbA at the end is for reference
only, as receiving stations do not acknowledge bulletins).


\subsection{NTS Radiograms}

APRS can be used to transport NTS radiograms. This uses the existing
APRS message format for backwards compatibility, by adding a 3-character
NTS format identifier Nx\ at the start of the APRS Message Text, as
follows:

N#\number\precedence\handling\originator\check\place\time\date
NA\address_line1\address_line2\address_line3\address_line4
NP\phone number
N1\line 1 of NTS message text
N2\line 2 of NTS message text
N3\line 3 of NTS message text
N4\line 4 of NTS message text
N5\line 5 of NTS message text
N6\line 6 of NTS message text
NS\Signature block
NR\Received from\date_time\sent_to\date_time

All of these fields comes from the ARRL NTS Radiogram form and are
described in the NTS handbook.

Each message line is addressed to the same station.
The N#\, NA\ and NR\ lines are multiple fields combined for APRS
transmission efficiency. The backslash separator is used so that conventional
forward slashes may be embedded in messages. (The backslash does not exist
in the RTTY or CW alphabets, so it therefore cannot appear in an NTS
radiogram).

Each line may be up 67 characters long, including the 3-character NTS
format identifier. Lines in excess of 67 characters will be truncated.
There is a maximum of 6 lines of NTS message text.

Note: The N#\, NA\, NS\ and NR\ fields are required. The others are
optional.
Serialization of each line is handled by the normal APRS Message ID
{xxxxx.

An APRS application is not required to understand or generate these
messages. The information can be read and understood in the normal
message display.

\subsection{Obsolete Bulletin and Announcement Format}

Some stations transmit bulletins and announcements without the : APRS
Data Type Identifier. This format is obsolete. Some software may still
decode such data as a bulletin or announcement, but it should now be
interpreted as a Status Report.



\section{Bulletin and Announcement Implementation Recommendations}


Bulletins and announcements are seen as a way for all participants in an
event/emergency/net to see all common information posted to the group. In
this sense they are visualized as a mountain-top billboard or a bulletin board
on the wall of an Emergency Operations Control Center.

Information that everyone must see is to be posted there. Information is
added and removed. Space is limited. Only so much information can be
posted before it becomes too busy for anyone to see everything. Thus things
are supposed to be posted, updated, and cleared to keep the big billboard
uncluttered and current with what everyone needs to know at the present
time. It should not be cluttered with obsolete information.

This can be implemented in an APRS display system as a “Bulletin Screen”.
Everyone has this screen, and anyone can post or update lines on this screen.
At any instant, everyone in the network sees exactly the same screen.
Everything is arranged and displayed in exactly the same way. Thus,
everyone, everywhere is looking at the same mountain-top billboard or
bulletin board. There is no ambiguity as to who sees what information, in
what order at what time.

To make this work, a number of issues should be considered:
•

•

•

•

•

Sorting: Bulletins/Announcements are almost always multi-line, and
may arrive out of sequence. They must be sorted before presentation on
the Bulletin Screen, and re-sorted if necessary when each new line
arrives. This includes sorting by originating callsign and Bulletin/
Announcement ID.

Replacement: Stations sending bulletins/announcements can send new
lines to replace lines sent earlier, re-using the original Bulletin/
Announcement IDs. (Note: It is only necessary to re-send replacement
lines. It is not necessary to re-send the whole bulletin/announcement).
Receipt of a new line with the same Bulletin/Announcement ID as one
already received from the same station should result in the existing line
being overwritten (replaced).

Clearing: A user should be able to clear any or all of the bulletins/
announcements from the Bulletin Screen once he has read them. Any
bulletins/announcements that are still valid will re-appear in due course
because of the way they are redundantly re-transmitted.
Alerts: On receipt of any new or replacement line for the Bulletin
Screen, an alarm should be sounded and re-sounded periodically until the
user acknowledges it. Thus, this vital information is “pushed” to the
operator. There is no excuse for not being aware of the current bulletin/
announcement state — this is important in the hurried and crisis-laden
scenario of an APRS event.

Logging: Old bulletins/announcements should be logged in sequential

APRS log files in case they are subsequently needed.
