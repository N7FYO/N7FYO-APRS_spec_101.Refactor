\chapter{Chapter 15: Station Capabilities, Queries and Responses}

\section{Station Capabilities}

A station may define a set of one or more attributes of the station, known as
Station Capabilities. The station transmits its capabilities in response to an
IGATE query (see below), using the < Data Type Identifier.
Each capability is a TOKEN or a TOKEN=VALUE pair. More than one
capability may be on a line, with each capability separated by a comma.
Currently defined capabilities include:
IGATE,MSG_CNT=n,LOC_CNT=n

where IGATE defines the station as an IGate, MSG_CNT is the number of
messages transmitted, and LOC_CNT is the number of “local” stations (those
to which the IGate will pass messages in the local RF network).

Queries and
Responses

There are two types of APRS queries. One is general to all stations and the
other is in a message format directed to a single individual station.
Queries always begin with a ?, are one-time transmissions, do not have a
message identifier and should not be acknowledged. Similarly the responses
to queries are one-time transmissions that also do not have a message
identifier, so that they too are not acknowledged.

Each query contains a Query Type (in upper-case). The following Query
Types and expected responses are supported:

Query Type
APRS
APRSD
APRSH
APRSM
APRSO
APRSP
APRSS
APRST or
PING?
IGATE
WX

Query
General — All stations query
Directed — Query an individual station for
stations heard direct
Directed — Query if an individual station has
heard a particular station
Directed — Query an individual station for
outstanding unacknowledged or undelivered
messages
Directed — Query an individual station for its
Objects
Directed — Query an individual station for its
position
Directed — Query an individual station for its
status
Directed — Query an individual station for a
trace (i.e. path by which the packet was heard)
General — Query all Internet Gateways
General — Query all weather stations

Response
Station’s position and status
List of stations heard direct
Position of heard station as an APRS Object, plus
heard statistics for the last 8 hours
All outstanding messages for the querying station

Station’s Objects
Station’s position
Station’s status
Route trace
IGate station capabilities
Weather report (and the station’s position if it is
not included in the Weather Report)


If a queried station has no relevant information to include in a response, it
need not respond.
A queried station should ignore any query that it does not recognize.

The format of a general query is as follows:

General Queries

General Query Format
Target Footprint

? Query

?

Type
Bytes:

1

n

1

Lat

,

Long

,

Radius

n

1

n

1

4

Examples
Query
?APRS?
General query, with standard posit and status reply.
?APRS?V
V34.02,-117.15,0200
General query for stations within a target footprint
of radius 200 miles centered on 34.02 degrees
north, 117.15 degrees west, with standard posit
and status reply. (Note the leading space in the
latitude, as its value is positive, see below).
?IGATE?
General query for IGate stations, with a Station
Capabilities reply.
?WX?
Query for weather stations, with a standard
Weather Report reply (without a position),
followed by a standard posit.

Typical Response
/092345z4903.50N/07201.75W>
>092345zNet Control Center
/3402.78N11714.02W>Digi on low power

<IGATE,MSG_CNT=43,LOC_CNT=14

_10090556c220s004g005t077…
/090556z4903.50N/07201.75W>

In the case of an ?APRS? query for stations within a particular target
footprint, the latitude and longitude parameters are in floating point degrees
(not in APRS lat/long position format).
•

North and east coordinates are positive values, indicated by a leading V
(space).

•

South and west coordinates are negative values.

•

The radius of the footprint is in miles, expressed as a fixed 4-digit
number in whole miles.

All stations inside the specified coverage circle should respond with a
Position Report and a Status Report.


Queries addressed to individual stations are in APRS message format (except
that they never include a message identifier). The addressee is the callsign of
the station being queried.

Directed Station
Queries

The message text is the Query Type. This is followed optionally by another
callsign — this callsign does not need filler spaces as it is at the end of the
data.

Directed Station Query Format
Addressee

:
Bytes:

1

: ?
9

1

1

Query
Type

Callsign of
Heard Station

5

0-9

Examples
Query

Typical Response

:KH2ZVVVVV:?APRSD
A query asking KH2Z what stations he
has heard direct.

:N8URVVVVV:Directs=VWA1LOUVWD5IVD…

VVVV
:KH2ZVVVVV:?APRSHVN0QBFVVVV
A query asking for the number of times
N0QBF was heard in each of the last
8 hours. (Note the trailing spaces in the
callsign following APRSH, padding the
callsign to 9 characters).

:N8URVVVVV:N0QBFVHEARD:V1V3V2V.V.V4V5V6

:KH2ZVVVVV:?APRSM
A query asking KH2Z for any
unacknowledged or undelivered
messages for him. KH2Z responds
with all such messages.

:N8URVVVVV:Testing{003

:KH2ZVVVVV:?APRSO
;LEADERVVV*092345z4903.50N/07201.75W>
A query asking for KH2Z’s APRS Objects.
:KH2ZVVVVV:?APRSP
A query asking for KH2Z’s position.

/092345z4903.50N/07201.75W>

:KH2ZVVVVV:?APRSS
A query asking for KH2Z’s status.

>092345zNet Control Center

:KH2ZVVVVV:?APRST
A query asking KH2Z for a trace of the
route taken to reach him.

:N8URVVVVV:KH2Z>APRS,DIGI1,WIDE*:

:KH2ZVVVVV:?PING?
The same query, using PING instead
of APRST.

:N8URVVVVV:KH2Z>APRS,DIGI1,WIDE*:

