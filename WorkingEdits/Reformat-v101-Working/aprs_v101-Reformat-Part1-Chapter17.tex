\chapter{Chapter 17: Network Tunneling and Third-Party Digipeating}

\section{Third-Party Networks}

APRS provides a mechanism for formatting packets that are to be transported
through third-party (i.e. non AX.25) networks, such as the Internet, an
Ethernet LAN or a direct wire connection.
These networks do not understand APRS source, destination and digipeater
addresses, so it is necessary to send them as data, along with the original data
being transmitted.

\section{Source Path Header}

Prior to sending an APRS packet into the third-party network, the APRS
address path is prepended to the Data Type Identifier and the rest of the
original data.
The prepended address path is known as the Source Path Header. It consists
of the source, destination and digipeater callsigns, with associated SSIDs.
The main purpose of introducing the Source Path Header is to allow
receiving stations on the far side of the third-party network to identify the
sender — this is needed when acknowledging receipt of a message, for
example. Knowledge of the source path is also useful in diagnosing network
problems.

Data with Source Path Header

Bytes:

Source
Path
Header

Data
Type
ID

Rest of the original data

n

1

n

The Source Path Header may be in either of two formats, known as the
“TNC-2” format and the “AEA” format (so called because when TNC-2 or
AEA-compatible TNCs are operating in terminal MONitor mode they
automatically produce headers in these formats).
The APRS Working Group has agreed to move towards standardization on
the “TNC-2” format in future implementations.
In most cases, AEA TNCs will produce Source Path Headers in “TNC-2”
format when BBSMSGS is set to ON.

The formats of these headers are as follows:
Source Path Header — “TNC-2” Format
An asterisk follows the digipeater callsign heard.

Bytes:

Source
Callsign
(-SSID)

>

Destination
Callsign
(-SSID)

1-9

1

1-9

0-8 Digipeaters
Digipeater
Callsign
(-SSID)(*)

,

0-81

:
1

Example

WB4APR-14>APRS,RELAY*,WIDE:
(WIDE digipeater “unused”)
Source Path Header — “AEA” Format
An asterisk follows the source or digipeater callsign heard.
Source
Callsign
(-SSID)(*)
Bytes:

1-10

0-8 Digipeaters

>

Digipeater
Callsign
(-SSID)(*)
0-81

>

Destination
Callsign
(-SSID)

:

1

1-9

1

Example
WB4APR-14>RELAY*>WIDE>APRS:
(WIDE digipeater “unused”)

In both formats, the SSID may be omitted if it is –0.
In both formats, the callsign of the digipeater from which the incoming
packet was heard is indicated with an asterisk. (Alternatively, for “AEA”
format only, the asterisk will follow the source callsign if the packet was
heard direct from there).

Any digipeaters following the callsign of the station from which the packet
was heard are termed “unused”. These unused digipeaters are stripped out
when building a Third-Party Header (see below).


After a packet emerges from a third-party network, the receiving gateway
station modifies it (by inserting a } Third-Party Data Type Identifier and
modifying the Source Path Header) before transmitting it on the local APRS
network.

Third-Party Header

The modified Source Path Header is called the Third-Party Header.
Third-party Format

Bytes:

}

Third-Party
Header

Rest of the original data

1

n

n

In a similar way to the Source Path Header, The Third-Party Header can be
in either of two formats: “TNC-2” or “AEA” format.
Third Party Header — “TNC-2” format
Source Path
Header
(without “unused”
digipeaters, * or :)
Bytes:

3-99

,
1

Third-Party
Network
Identifier
(“callsign”)
1-9

,

Callsign of
Receiving
Gateway Station
(-SSID)

1

1-9

* :
1

1

Example
WB4APR-14>APRS,RELAY,TCPIP,G9RXG*:
Third Party Header — “AEA” format
Source Path
Header
(without “unused”
digipeaters, destination,
* or :)

Third-Party
Network
Identifier
(“callsign”)

2-90

1-9

Bytes:

>

Callsign of
Receiving
Gateway Station
(-SSID)

1

1-9

* >

1

1

Destination
Callsign from
Source Path
Header
(-SSID)

:

1-9

1

Example
WB4APR-14>RELAY>TCPIP>G9RXG*>APRS:

In both cases, the “unused” digipeater callsigns (i.e. those digipeater
callsigns after the asterisk) in the original Source Path Header are stripped
out. The asterisk itself is also stripped out of the Source Path Header.
Then two additional callsigns are inserted:

\begin{itemize}

\item The Third-Party Network Identifier (e.g. TCPIP). This is a dummy
“callsign” that identifies the nature of the third-party network.

\item The callsign of the receiving gateway station, followed by an asterisk.

\end{itemize}

\section{Action on Receiving a Third-Party packet}

When another station receives a third-party packet, it can extract the callsign
of the original sending station from the Third-Party Header, if it is needed to
acknowledge receipt of a message.

The other addresses in the Third-Party Header may be useful for network
diagnostic purposes.

An Example of
Sending a Message
through the Internet

The Scenario:
•
•

WB4APR-14 wants to send a message via the Internet to G3NRW.
The nearest Internet gateway to WB4APR-14 is K4HG, reachable via a
RELAY,WIDE path.

•

The nearest Internet gateway to G3NRW is G9RXG.

The Process:
•

In the normal way, WB4APR-14 builds a message packet that contains:
:G3NRWV
V V V V :Hi Ian{001

•

WB4APR-14 transmits the packet via his UNPROTO path RELAY,WIDE.

•

The Internet gateway K4HG happens to receive this packet from the RELAY
digipeater in the path.

•

K4HG builds a new packet that contains the source path and the original
message:
WB4APR-14>APRS,RELAY*,WIDE::G3NRWV
V V V V :Hi Ian{001

•

K4HG sends this packet (using telnet) to an APRServer on the Internet.

•

All Internet gateways throughout the world that are connected to the
APRServe network (including G9RXG) receive the packet.

•

G9RXG converts the packet into a Third-Party packet:

}WB4APR-14>APRS,RELAY,TCPIP,G9RXG*::G3NRWV
V V V V :Hi Ian{001

Note that the WIDE digipeater was stripped out of the header because it was
unused.
•

G9RXG transmits the packet over the local APRS network.

•

G3NRW receives the packet, strips out the Third-Party Header, and discovers
that the packet contains a message for him. From the header, G3NRW then
establishes that the acknowledgement is to go back to WB4APR-14.
