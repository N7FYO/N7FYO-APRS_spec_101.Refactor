\chapter{Chapter 18: User-Defined Data Format}


The APRS protocol defines many different data formats, but it cannot
anticipate every possible data type that programmers may wish to send. The
User-Defined data format is designed to fill these gaps. Under this system,
program authors are free to send data in any format they choose.
The data in the AX.25 Information field consists of a three-character header:
{

APRS Data Type Identifier.

U

A one-character User ID.

X

A one-character user-defined packet type.

The APRS Working Group will issue User IDs to program authors who
express a need.
[Keep in mind there is a limited number of available User IDs, so please do
not request one unless you have a true need. The Working Group may require
an explanation of your need prior to issuing a character. If only one or two
data formats are needed, those may be issued from a User ID pool].
For experimentation, or prior to being issued a User ID, anyone may utilize
the User ID character of { without prior notification or approval (i.e. packets
beginning with {{ are experimental, and may be sent by anyone).
Important Note: Although there is no restriction on the nature of user-defined data, it is highly recommended that it is represented in printable 7-bit
ASCII character form.

User-Defined Data Format

Bytes:

User
ID

User-Defined
Packet Type

{

U

X

1

1

1

Examples
{Q1qwerty
{{zasdfg

User-defined data (printable ASCII recommended)
n

User ID = Q, User-defined packet type = 1.
User ID undefined (experimental), User-defined packet type = z.

This is envisioned as a way for authors to experiment and build in features
specific to their programs, without the danger of a non-standard packet
crashing other authors’ programs. In keeping with the spirit of the APRS
protocol, authors are encouraged to make these formats public. The APRS
Working Group will maintain a web site defining all of the assigned User
IDs, and either the packet formats provided by the author, or links to their
own web sites which define their formats.
Generally, all formats using this method will be considered optional. No
program is required to decode any of these packets, and must ignore any it
does not decode. However, it is possible that in the future some of these
formats may prove to be of sufficient utility and interest to the entire APRS
community that they will be specifically included in future versions of the
APRS protocol.
