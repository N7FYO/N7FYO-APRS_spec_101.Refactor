\chapter{Chapter 19: Other Packets}

\section{Invalid Data or Test Data Packets}

To indicate that a packet contains invalid data, or test data that does not
conform to any standard APRS format, the , Data Type Identifier is used.
For example, the Mic-E unit will generate such a packet if it detects that a
received GPS sentence is not valid.

Invalid Data / Test Data Format

Bytes:

,

Invalid Data or Test Data

1

n

Example
,191146,V,4214.2466,N,07303.5181,W,417.238,114.5,091099,14.7,W/GPS FIX
Invalid GPS data from a Mic-E unit. The unit has interpreted the V character in the
received sentence to mean the data is invalid, and has stripped out the $GPRMC header.

All Other Packets

Packets that do not meet any of the formats described in this document are
assumed to be non-APRS beacons. Programs can decide to handle these, or
ignore them, but they must be able to process them without ill effects.
APRS programs may treat such packets as APRS Status Reports. This allows
APRS to accept any UI packet addressed to the typical beacon address to be
captured as a status message. Typical TNC ID packets fall into this category.
Once a proper Status Report (with the APRS Data Type Identifier >) has
been received from a station it will not be overwritten by other non-APRS
packets from that station.
