\chapter{Chapter 20: APRS Symbols}

\section{Three Methods}

There are three methods of specifying an APRS symbol (display icon):
•

In the AX.25 Information field.

•

In the AX.25 Destination Address.

•

In the SSID of the AX.25 Source Address.

The preferred method is to include the symbol in the Information field.
However, where this is not possible (for example, in stand-alone trackers
with no means of introducing the symbol into the Information field), either of
the other two methods may be used instead.

The Symbol Tables

There are two APRS Symbol Tables:
•

Primary Symbol Table

•

Alternate Symbol Table

See Appendix 2 for a full listing of these tables.
The essential difference between the Primary and Alternate Symbol Tables is
that some of the symbols in the Alternate Symbol Table can be overlaid with
an alphanumeric character. For example, a “car” icon in the Alternate
Symbol Table could be overlaid with the digit “3”, to indicate it is car #3.
Symbols capable of taking an overlay are marked as [with overlay].
None of the symbols in the Primary Symbol Table can be overlaid.
In the tables, each symbol is coded in three ways:
•

/$ or \$ — for symbols in the Information field.

•

GPSxyz — for generic Destination addresses containing symbols.

•

GPSCnn or GPSEnn — another form of generic Destination addresses
containing systems.

In addition, 15 of the symbols in the Primary Symbol Table have an
associated SSID (e.g. a small aircraft has SSID -7). The SSID is intended for
use in the AX.25 Source Address of stand-alone trackers which have no other
means of specifying the symbol.

Symbols in the
AX.25 Information
Field

A symbol in the AX.25 Information field is a combination of a one-character
Symbol Table Identifier and a one-character Symbol Code.
For example, in the Position Report:


@092345z4903.50N/07201.75W>088/036…

the forward slash / is the Symbol Table Identifier and the > character is the
Symbol Code (in this case representing a “car” icon) from the selected table.
The Symbol Table Identifier character selects one of the two Symbol Tables,
or it may be used as single-character (alpha or numeric) overlay, as follows:
Symbol Table
Identifier

Selected Table or Overlay Symbol

/

Primary Symbol Table (mostly stations)

\

Alternate Symbol Table (mostly Objects)

0-9

Numeric overlay. Symbol from Alternate
Symbol Table (uncompressed lat/long
data format)

a-j

Numeric overlay. Symbol from Alternate
Symbol Table (compressed lat/long data
format only). i.e. a-j maps to 0-9

A-Z

Alpha overlay. Symbol from Alternate
Symbol Table

In the generic case, a symbol from the Primary Symbol Table is represented
as the character-pair /$, and a symbol from the Alternate Symbol Table as
\$.

Overlays with
Symbols in the
AX.25 Information
Field

Where the Symbol Table Identifier is 0-9 or A-Z (or a-j with compressed
position data only), the symbol comes from the Alternate Symbol Table, and
is overlaid with the identifier (as a single digit or a capital letter).
For example, in the uncompressed Position Report:
@092345z4903.50N307201.75W>…

the digit 3 following the latitude will cause the number “3” to be overlaid on
top of the “car” icon (Note: Because the symbol is overlaid, the > Symbol
Code here comes from the Alternate Symbol Table).
Similarly, to overlay a “car” icon with the letter “B” in a compressed
Position Report, the report will look something like:
=BL!!<*e7 >7P[

However, in a compressed Position Report, it is not permissible to use a
numeric Symbol Table Identifier (0-9) — compressed positions never start
with a digit. If a numeric overlay is required, the report must use a lower-case
letter instead (in the range a-j) as the Symbol Table Identifier. The lowercase letter is then mapped to the digits 0-9 (i.e. a=0, b=1, c=2, d=3 etc).
Thus, in the compressed Position Report:


=d5L!!<*e7 >7P[

the letter d maps to overlay character “3”.
As noted above, not all symbols from the Alternate Symbol Table may be
overlaid in this way — those that can be overlaid are marked as [with overlay]
in Appendix 2. This means that they are capable of taking an overlay, but
they do not necessarily need to have one. Thus, for example, the following
report uses the car symbol from the Alternate Symbol Table, but does not
display an overlay:
@092345z4903.50N\07201.75W>…

Symbols in the
AX.25 Destination
Address

Where it is not possible to include a symbol in the Information field, the
symbol may be specified in the AX.25 Destination Address instead, using the
following generic destination addresses: GPSxyz, GPSCnn, GPSEnn,
SPCxyz and SYMxyz.
The characters xy and nn refer to entries in the APRS Symbol Tables. For
example, from the Primary Symbol Table, a tracker could use the Destination
Address GPSMVV or GPS30 to specify a “car” icon.
The character z specifies the overlay character (where permitted), or is a V
(space) — the space is a filler character, as all AX.25 addresses must be
exactly 6 characters long.
The GPS/SPC/SYMxyV and GPSCnn/GPSEnn addresses can be used
interchangeably. Thus, for example, GPSBMV , SPCBMV , SYMBMV and
GPSC12 all specify a “Boy Scouts” icon (from the Primary Symbol Table),
and GPSOMV , SPCOMV , SYMOMV and GPSE12 all specify a “Girl Scouts”
icon (from the Alternate Symbol Table).

Overlays with
Symbols in the
AX.25 Destination
Address

If the z character in a GPSxyz, SPCxyz or SYMxyz address is not a space,
it specifies an alphanumeric overlay character, in the range 0-9 or A-Z.
Overlays can only be used with symbols from the Alternate Symbol Table
marked with the legend [with overlay].
For example, if the “car” icon is to be overlaid with a digit “3”, the
Destination Address will be GPSNV3.
However, even if the address is overlay-capable, it is not actually necessary
to specify an overlay; e.g. GPSNVV.
GPSCnn and GPSEnn symbols can not have overlays.


Symbol in the
Source Address
SSID

Where it is not possible to include a symbol in the Information field or in the
Destination Address, the symbol may be specified in the SSID of the Source
Address instead:
SSID-Specified Icons in the AX.25 Source Address Field
SSID
-0
-1
-2
-3
-4
-5
-6
-7

Symbol Precedence

Icon

SSID
-8
-9
-10
-11
-12
-13
-14
-15

[no icon]
Ambulance
Bus
Fire Truck
Bicycle
Yacht
Helicopter
Small Aircraft

Icon
Ship (power boat)
Car
Motorcycle
Balloon
Jeep
Recreational Vehicle
Truck
Van

APRS packets should not contain more than one symbol. However, it is
conceivably possible to (erroneously) construct a packet containing up to
three different symbols.
For example:

Symbol

Source
Address SSID

Destination
Address

Information Field

G3NRW-7

GPSMV

!0123.45N/01234.56Wj

Small Aircraft

Car

Jeep

In such a situation:
•

The symbol in the Information field takes precedence over any other
symbol.

•

If there is no symbol in the Information field, the symbol in the
Destination Address takes precedence over the symbol in the Source
Address SSID.

