\chapter{Chapter 4: APRS Data in the AX.25 Destination and Source Address Fields}


\section{The AX.25 Destination Address Field}

The AX.25 Destination Address field can contain 6 different types of APRS
information:

\begin{itemize}

\item A generic APRS address.

\item A generic APRS address with a symbol.

\item An APRS software version number.

\item Mic-E encoded data.

\item A Maidenhead Grid Locator (obsolete).

\item An Alternate Net (ALTNET) address.

\end{itemize}

In all of these cases, the Destination Address SSID may specify a generic
APRS digipeater path.

\section {Generic APRS Destination Addresses}

APRS uses the following generic beacon-style destination addresses:

% Start Chart

\begin{tabbing}
\= 123456789 \=123456789 \=123456789 \=123456789 \=123456789 \=123456789 \=123456789 \kill
\> AIR* $\dagger$  \> ALL*   \> AP*   \> BEACON \> CQ*  \> GPS*   \> DF*   \\
\> DGPS*  \> DRILL* \> DX*   \> ID*    \> JAVA* \> MAIL* \> MICE* \\
\> QST*   \> QTH*   \> RTCM* \>SKY*    \> SPACE* \> SPC*  \> SYM* \\
\> TEL*   \> TEST*  \> TLM*  \> WX*    \> ZIP* $\dagger$ \\

\end{tabbing}

The asterisk is a wildcard, allowing the address to be extended (up to a total
of 6 alphanumeric characters). Thus, for example, WX1, WX12 and WX12CD
are all valid APRS destination addresses.

$\dagger$ The AIR* and ZIP* addresses are being phased out, but are needed at
present for backward compatibility.

All of these addresses have an SSID of –0. Non-zero SSIDs are reserved for
generic APRS digipeating.

These addresses are copied by everyone. All APRS software must accept
packets with these destination addresses.

The address GPS (i.e. the 3-letter address GPS, not GPS*) is specifically
intended for use by trackers sending lat/long positions via digipeaters which
have the capability of converting positions to compressed data format.

The addresses DGPS and RTCM are used by differential GPS correction
stations. Most software will not make use of packets using this address, other
than to pass them on to an attached GPS unit.

The address SKY is used for Skywarn stations.

Packets addressed to SPCL are intended for special events. APRS software
can display such packets to the exclusion of all others, to minimize clutter on
the screen from other stations not involved in the special event.

The addresses TEL and TLM are used for telemetry stations.

\section{Generic APRS Address with Symbol}

APRS uses several of the above-listed generic addresses in a special way, to
specify not only an address but also a display symbol. These special
addresses are \textsf {GPSxyz, GPSCnn, GPSEnn, SPCxyz} and \textsf {SYMxyz}, and are
intended for use where it is not possible to include the symbol in the AX.25
Information field.

The GPS addresses above are for general use.

The SPC addresses are intended for special events.

The SYM addresses are reserved for future use.

The characters \textsf{xy} and \textsf{nn} refer to entries in the APRS Symbol Tables. The
character \textsf{z} specifies a symbol overlay. See Chapter 20: APRS Symbols and
Appendix Two for more information.

\section{APRS Software Version Number}

The AX.25 Destination Address field can contain the version number of the
APRS software that is running at the station. Knowledge of the version
number can be useful when debugging.

The following software version types are reserved (xx and xxx indicate a
version number):

\begin{itemize}

\item APCxxx APRS/CE, Windows CE
\item APDxxx Linux aprsd server
\item APExxx PIC-Encoder
\item APIxxx Icom radios (future)
\item APICxx ICQ messaging
\item APKxxx Kenwood radios
\item APMxxx MacAPRS
\item APPxxx pocketAPRS
\item APRxxx APRSdos
\item APRS older versions of APRSdos
\item APRSM older versions of MacAPRS
\item APRSW older versions of WinAPRS
\item APSxxx APRS+SA
\item APWxxx WinAPRS
\item APXxxx X-APRS
\item APYxxx Yaesu radios (future)
\item APZxxx Experimental

\end{itemize}

This table will be added to by the APRS Working Group.
For example, a station using version 3.2.6 of MacAPRS could use the
destination callsign \textsf{APM326}.

The Experimental destination is designated for temporary use only while a
product is being developed, before a special APRS Software Version address
is assigned to it.

\section {Mic-E Encoded Data}

Another alternative use of the AX.25 Destination Address field is to contain
Mic-E encoded data. This data includes:

\begin{itemize}

\item The latitude of the station.

\item A West/East Indicator and a Longitude Offset Indicator (used in
longitude computations).

\item A Message Code.

\item The APRS digipeater path.

\end{itemize}
This data is used with associated data in the AX.25 Information field to
provide a complete Position Report and other information about the station
(see Chapter 10: Mic-E Data Format).

\section {Maidenhead Grid Locator in Destination Address}

The AX.25 Destination Address field may contain a 6-character Maidenhead
Grid Locator. For example: \textsf{IO91SX}. This format is typically used by meteor
scatter and satellite operators who need to keep packets as short as possible.

This format is now obsolete.

\section{Alternate Nets}

Any other destination address not included in the specific generic list or the
other categories mentioned above may be used in Alternate Nets (ALTNETs)
by groups of individuals for special purposes. Thus they can use the APRS
infrastructure for a variety of experiments without cluttering up the maps and
lists of other APRS stations. Only stations using the same ALTNET address
should see their data.

\section{Generic APRS Digipeater Path}


The SSID in the Destination Address field of all packets is coded to specify
the APRS digipeater path.

If the Destination Address SSID is –0, the packet follows the standard AX.25
digipeater (“VIA”) path contained in the Digipeater Addresses field of the
AX.25 frame.

If the Destination Address SSID is non-zero, the packet follows one of 15
generic APRS digipeater paths.


The SSID field in the Destination Address (i.e. in the 7th address byte) is
encoded as follows:



\subsubsection*{APRS Digipeater Paths in Destination Address SSID}

\begin{multicols}{2}

      
  \begin{tabular}{|c|c|}
    \hline
    SSID & Path \\       \hline
    -0 & Use VIA path \\ \hline
    -1 & WIDE1-1 \\      \hline
    -2 & WIDE2-2 \\ \hline
    -3 & WIDE3-3 \\ \hline
    -4 & WIDE4-4 \\ \hline
    -5 & WIDE5-5 \\ \hline
    -6 & WIDE6-6 \\ \hline
    -7 & WIDE7-7 \\ \hline
  \end{tabular}
  
  \begin{tabular}{|c|c|}
    \hline
    SSID & Path \\       \hline
    -8 & North path \\ \hline
    -9 & South path \\      \hline
    -10 & East path \\ \hline
    -11 & West path \\ \hline
    -12 & North path + WIDE \\ \hline
    -13 & South path + WIDE \\ \hline
    -14 & East path + WIDE \\ \hline
    -15 & West path + WIDE \\ \hline
  \end{tabular}
\end{multicols}


\section{The AX.25 Source Address SSID to specify Symbols}

The AX.25 Source Address field contains the callsign and SSID of the
originating station. If the SSID is –0, APRS does not treat it in any special
way.

If, however, the Source Address SSID is non-zero, APRS interprets it as a
display icon. This is intended for use only with stand-alone trackers where
there is no other method of specifying a display symbol or a destination
address (e.g. MIM trackers or NMEA trackers).

For more information, see Chapter 20: APRS Symbols.
