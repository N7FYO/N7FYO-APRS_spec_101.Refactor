\chapter {Chapter 5: APRS Data in the AX.25 Information Field}

\section {Generic Data Format}

In general, the AX.25 Information field can contain some or all of the
following information:
\begin{itemize}
\item  APRS Data Type Identifier
\item  APRS Data
\item  APRS Data Extension
\item  Comment
\end{itemize}

% begin chart

\begin{tabular}{|l|c|c|c|c|}
  \hline
  \multicolumn{5}{|l|}{\bf\it{Generic APRS Information Field}} \\
  \hline
  & Data Type ID & APRS Data & APRS Data Extension & Comment \\
  \hline 
  Bytes: & 1            & n         & 7                   & n \\
  \hline
\end{tabular}


% end chart

\section{APRS Data Type Identifier}

Every APRS packet contains an APRS Data Type Identifier (DTI). This
determines the format of the remainder of the data in the Information field, as
follows:

% begin chart
% top label figure: APRS Data Type Identifiers


\begin{table}[htbp]

\caption{\bf{APRS Data Type Identifiers}}
\begin{tabular} {|l|l|}
    \hline
    \bf\it{Ident} & \bf\it{Data Type} \\
    \hline
     0x1c & Current Mic-E Data (Rev 0 beta) \\
     \hline
     0x2d & Old Mic-E Data (Rev 0 beta) \\
     \hline
     $!$ & Position without timestamp (no APRS messaging), or Ultimeter 2000 WX Station \\
     \hline
     \textquotedbl     & [Unused] \\     
     \hline

     \# & Peet Bros U-II Weather Station \\
     \hline
     \$ & Raw GPS data or Ultimeter 2000 \\
     \hline
     \% & Agrelo DFJr / MicroFinder \\
     \hline
     \& & [Reserved \textemdash Map Feature] \\
     \hline
     \textquotesingle & Old Mic-E Data (but \it{Current} data for TM-D700) \\
     \hline
     ( & \it{[Unused]} \\
     \hline
     ) & Item \\
     \hline
     * & Peet Bros U-II Weather Station \\
     \hline
     + & \it{[Reserved} \textemdash Shelter data with time] \\
     \hline
     , & Invalid data or test data \\
     \hline
     - & \it{[Unused]} \\
     \hline
     . & \it{[Reserved \textemdash Space weather]} \\
     \hline
     / & Position with timestamp (no APRS messaging) \\
     \hline
     0-9 & \it{[Do not use]} \\
     \hline
     : & Message \\
     \hline
     ; & Object \\
     \hline
\end{tabular}
\end{table}


\begin{table}[htbp]
\caption{\bf{APRS Data Type Identifiers, Continued}}
  
  \begin{tabular} {|l|l|}
    \hline
    \it{Ident} & \it{Data Type} \\
    \hline
     < & Station Capabilities \\
     \hline
     = & Position without timestamp (with APRS messaging) \\
     \hline
     > & Status \\
     \hline
     ? & Query \\
     \hline
     @ & Position with timestamp (with APRS messaging) \\
     \hline
     A-S & \it{[Do not use]} \\
     \hline
     T & Telemetry data \\
     \hline
     U-Z & \it{[Do not use]} \\
     \hline
     [ & Maidenhead grid locator beacon (obsolete) \\
       \hline
       \textbackslash & \it{[Unused]} \\
       \hline
     ] & \it{[Unused]} \\
       \hline
       \^{} & \it{[Unused]} \\
       \hline
       \_{} & Weather Report (without position) \\
       \hline
       \textasciigrave & Current Mic-E Data (not used in TM-D700) \\
       \hline
       a-z & \it{[Do not use]} \\
       \hline
       \{ & User-Defined APRS packet format \\
       \hline
       | & \it{[Do not use — TNC stream switch character]} \\
       \hline
       \} & Third-party traffic \\
       \hline
       \textasciitilde{} & \it{[Do not use — TNC stream switch character]} \\
       \hline
       
       
  \end{tabular}

  
\end{table}
  

% end chart




\textbf{Note:} There is one exception to the requirement for the Data Type Identifier
to be the first character in the Information field — this is the Position without
Timestamp (indicated by the ! DTI). The ! character may occur anywhere
up to and including the 40th character position in the Information field. This
variability is required to support X1J TNC digipeaters which have a string of
unmodifiable text at the beginning of the field.

% footnote that this exception was later removed in spec version 1.1 (check)

\textbf{Note:} The Kenwood TM-D700 radio uses the ' DTI for current Mic-E data.
The radio does not use the ‘ DTI.

\clearpage

\section{APRS Data and Data Extension}



There are 10 main types of APRS Data:

\begin {itemize}

\item Position
\item Direction Finding
\item Objects and Items
\item Weather
\item Telemetry
\item Messages, Bulletins and Announcements
\item Queries
\item Responses
\item Status
\item Other

\end{itemize}

Some of this data may also have an APRS Data Extension that provides
additional information.

The APRS Data and optional Data Extension follow the Data Type Identifier.
The table on the next page shows a complete list of all the different possible
types of APRS Data and APRS Data Extension.

\newpage
% Begin Table

% \begin{table}[htbp]
%   \caption{Possible APRS Data and Extension}

\renewcommand{\arraystretch}{1.1}
  \begin{tabular}{|r|l|l|}
    \hline
    \hline
    \multicolumn{3}{|c|}{Possible APRS Data and Extension} \\
    \hline
    \hline
    \hline
    & Possible APRS Data & Possible APRS Data Extension \\
    \hline

% Position Row
    
    Position & Time (DHM or HMS) & Course and Speed \\
    & Lat/long coordinates & Power, Effective Antenna Height \\
    & & /Gain/Directivity \\
    & Compressed lat/long/course/speed/radio range/altitude & Pre-Calculated Radio Range \\
    & Symbol Table ID and Symbol Code & Omni DF Signal Strength \\
    & Mic-E longitude, speed and course, telemetry or status & Storm Data (in Comment field) \\
    & Raw GPS NMEA sentente & \\
    & Raw weather station data & \\
    \hline
    
% Direction Finding Row 

    Direction Finding & Time (DHM or HMS) & Course and Speed \\
    & Lat/long coordinates & Power, Effective Antenna Height \\
    & & /Gain/Directivity \\
    & Compressed lat/long/course/speed/radio range/altitude & Pre-Calculated Radio Range \\
    & Symbol Table ID and Symbol Code & Omni DF Signal Strength \\
    & & Bearing and \\
    & & Number/Range/Quality \\
    & & (in Comment field) \\
    \hline

% Objects and Items Row
    Objects and & Object Name & Course and Speed \\
    Items & Item name & Power, Effective Antenna Height/Gain/Directivity \\
    & Time  (DHM or HMS) & Pre-Calculated Radio Range \\
    & Lat/long coordinates & \\
    & Compressed lat/long/course/speed/radio range/altitude & Omni DF Signal Strength \\
    & Symbol Table ID and Symbol Code & Area Object \\
    & Raw weather station data & Storm Data (in Comment field) \\
    \hline
    
% Weather Row

    Weather & Time (MDHM) & Wind Direction and Speed \\
    & Lat/long coordinates & Storm Data (in Comment field) \\
    & Compressed lat/long/course/speed/radio range/altitude \\
    & Symbol Table ID and Symbol Code \\
    & Raw weather station data \\
    \hline

% Telemetry Row
    
    Telemetry & Telemetry (non Mic-E) & \\
    \hline

    % Messages, Bulletins and Announcements Row
    Messages, &  Addressee & \\
    Bulletins and & Message Text & \\
    Announcements & Message Identifier & \\
    & Message Acknowledgement & \\
    & Bulletin ID, Announcement ID & \\
    & Group Bulletin ID & \\
    \hline

% Queries Row
    Queries & Query Type & \\
    & Query Target Footprint & \\
    & Addressee (Directed Query) & \\
    \hline
    
% Responses Row
    & Position & Course and Speed \\
    & Object/Item & Power, Effective Antenna Height/Gain/Directivity \\
    & Weather & Pre-Calculated Radio Range \\
    & Status & Omni DF Signal Strength \\
    Responses & Message & Area Object \\
    & Digipeater Trace & Wind Direction and Speed \\
    & Stations Heard & \\
    & Heard Statistics & \\
    & Station Capabilities &  \\
    \hline
    
% Status Row

    Status & Time (DHM zulu) & \\
    & Status text & \\
    & Meteor Scatter Beam Heading/Power & \\
    & Maidenhead Locator (Grid Square) & \\
    & Altitude (Mic-E) & \\
    & E-mail message & \\
    \hline
 
    % Other Row
    Other & Third-Party forwarding & \\
    & Invalid Data/Test Data & \\
    \hline
    
  \end{tabular}
% \end{table}

  % Return to global array Stretch
\renewcommand{\arraystretch}{\arrystrchfactor}

\clearpage


\section{Comment Field}

In general, any APRS packet can contain a plain text comment (such as a
beacon message) in the Information field, immediately following the APRS
Data or APRS Data Extension.

There is no separator between the APRS data and the comment unless
otherwise stated.

The comment may contain any printable ASCII characters (except | and ~,
which are reserved for TNC channel switching).

The maximum length of the comment field depends on the report — details
are included in the description of each report.

In special cases, the Comment field can also contain further APRS data:

\begin{itemize}


\item Altitude in comment text (see Chapter 6: Time and Position Formats), or
in Mic-E status text (see Chapter 10: Mic-E Data Format).

\item Maidenhead Locator (grid square), in a Mic-E status text field (see
Chapter 10: Mic-E Data Format) or in a Status Report (see Chapter 16:
Status Reports).

\item Bearing and Number/Range/Quality parameters (/BRG/NRQ), in DF
reports (see Chapter 7: APRS Data Extensions).

\item Area Object Line Widths (see Chapter 11: Object and Item Reports).

\item Signpost Objects (see Chapter 11: Object and Item Reports).

\item Weather and Storm Data (see Chapter 12: Weather Reports).

\item Beam Heading and Power, in Status Reports (see Chapter 16: Status
Reports).

\end{itemize}

\section{Base-91 Notation}

Two APRS data formats use base-91 notation: lat/long coordinates in
compressed format (see Chapter 9) and the altitude in Mic-E format (see
Chapter 10).

Base-91 data is compressed into a short string of characters. All the
characters are printable ASCII, with character codes in the range 33–124
decimal (i.e. ! through |).

To compute the base-91 ASCII character string for a given data value, the
value is divided by progressively reducing powers of 91 until the remainder
is less than 91. At each step, 33 is added to the modulus of the division
process to obtain the corresponding ASCII character code.

For example, for a data value of 12345678:
\begin{verbatim}
12345678 / 91^3 = modulus 16, remainder 288542
288542 / 91^2 = modulus 34, remainder 6988
6988 / 91^1 = modulus 76, remainder 72
\end{verbatim}

The four ASCII character codes are thus 49 (i.e. 16+33), 67 (i.e. 34+33), 109
(i.e. 76+33) and 105 (i.e. 72+33), corresponding to the ASCII string 1Cmi.

\section{APRS Data Units}

For historical reasons there is some lack of consistency between units of data
in APRS packets — some speeds are in knots, others in miles per hour; some
altitudes are in feet, others in meters, and so on. It is emphasized that this
specification describes the units of data as they are transmitted on-air. It is
the responsibility of APRS applications to convert the on-air units to more
suitable units if required.

The default GPS earth datum is World Geodetic System (WGS) 1984.


