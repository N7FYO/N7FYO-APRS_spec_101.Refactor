\chapter {Chapter 5: APRS Data in the AX.25 Information Field}

\section {Generic Data Format}

In general, the AX.25 Information field can contain some or all of the
following information:
\begin{itemize}
\item  APRS Data Type Identifier
\item  APRS Data
\item  APRS Data Extension
\item  Comment
\end{itemize}

% begin chart

Generic APRS Information Field
Data
Type ID

APRS Data

APRS Data
Extension

Comment

1

n

7

n

Bytes:

% end chart

\section{APRS Data Type Identifier}

Every APRS packet contains an APRS Data Type Identifier (DTI). This
determines the format of the remainder of the data in the Information field, as
follows:

% begin chart

APRS Data Type Identifiers
Ident

Data Type

Ident

Data Type

0x1c

Current Mic-E Data (Rev 0 beta)

<

Station Capabilities

0x1d

Old Mic-E Data (Rev 0 beta)

=

Position without timestamp (with APRS
messaging)

!

Position without timestamp (no APRS
messaging), or Ultimeter 2000 WX Station

>

Status

"

[Unused]

?

Query

#

Peet Bros U-II Weather Station

@

Position with timestamp (with APRS messaging)

$

Raw GPS data or Ultimeter 2000

%

Agrelo DFJr / MicroFinder

&

[Reserved — Map Feature]

'

Old Mic-E Data (but Current data for TM-D700)

[

Maidenhead grid locator beacon (obsolete)

(

[Unused]

\

[Unused]

)

Item

]

[Unused]

*

Peet Bros U-II Weather Station

^

[Unused]

+

[Reserved — Shelter data with time]

_

Weather Report (without position)

,

Invalid data or test data

‘

-

[Unused]

.

[Reserved — Space weather]

{

User-Defined APRS packet format

/

Position with timestamp (no APRS messaging)

|

[Do not use — TNC stream switch character]

[Do not use]

}

Third-party traffic

:

Message

~

[Do not use — TNC stream switch character]

;

Object

0–9

A–S
T
U–Z

a–z



[Do not use]
Telemetry data
[Do not use]

Current Mic-E Data (not used in TM-D700)
[Do not use]



% end chart


\textbf{Note:} There is one exception to the requirement for the Data Type Identifier
to be the first character in the Information field — this is the Position without
Timestamp (indicated by the ! DTI). The ! character may occur anywhere
up to and including the 40th character position in the Information field. This
variability is required to support X1J TNC digipeaters which have a string of
unmodifiable text at the beginning of the field.

% footnote that this exception was later removed in spec version 1.1 (check)

\textbf{Note:} The Kenwood TM-D700 radio uses the ' DTI for current Mic-E data.
The radio does not use the ‘ DTI.

\section{APRS Data and Data Extension}



There are 10 main types of APRS Data:

\begin {itemize}

\item Position
\item Direction Finding
\item Objects and Items
\item Weather
\item Telemetry
\item Messages, Bulletins and Announcements
\item Queries
\item Responses
\item Status
\item Other

\end{itemize}

Some of this data may also have an APRS Data Extension that provides
additional information.

The APRS Data and optional Data Extension follow the Data Type Identifier.
The table on the next page shows a complete list of all the different possible
types of APRS Data and APRS Data Extension.



Possible APRS Data

Possible APRS Data Extension

Time (DHM or HMS)
Lat/long coordinates
Compressed lat/long/course/speed/radio range/altitude
Symbol Table ID and Symbol Code
Mic-E longitude, speed and course, telemetry or status
Raw GPS NMEA sentence
Raw weather station data

Course and Speed
Power, Effective Antenna Height/Gain/Directivity
Pre-Calculated Radio Range
Omni DF Signal Strength
Storm Data (in Comment field)

Time (DHM or HMS)
Lat/long coordinates
Compressed lat/long/course/speed/radio range/altitude
Symbol Table ID and Symbol Code

Course and Speed
Power, Effective Antenna Height/Gain/Directivity
Pre-Calculated Radio Range
Omni DF Signal Strength
Bearing and Number/Range/Quality
(in Comment field)

Objects and
Items

Object name
Item name
Time (DHM or HMS)
Lat/long coordinates
Compressed lat/long/course/speed/radio range/altitude
Symbol Table ID and Symbol Code
Raw weather station data

Course and Speed
Power, Effective Antenna Height/Gain/Directivity
Pre-Calculated Radio Range
Omni DF Signal Strength
Area Object
Storm Data (in Comment field)
Wind Direction and Speed
Storm Data (in Comment field)

Weather

Time (MDHM)
Lat/long coordinates
Compressed lat/long/course/speed/radio range/altitude
Symbol Table ID and Symbol Code
Raw weather station data

Position

Direction Finding

Telemetry
Messages,
Bulletins and
Announcements

Queries

Responses

Status

Other

Telemetry (non Mic-E)
Addressee
Message Text
Message Identifier
Message Acknowledgement
Bulletin ID, Announcement ID
Group Bulletin ID
Query Type
Query Target Footprint
Addressee (Directed Query)
Position
Object/Item
Weather
Status
Message
Digipeater Trace
Stations Heard
Heard Statistics
Station Capabilities

Course and Speed
Power, Effective Antenna Height/Gain/Directivity
Pre-Calculated Radio Range
Omni DF Signal Strength
Area Object
Wind Direction and Speed

Time (DHM zulu)
Status text
Meteor Scatter Beam Heading/Power
Maidenhead Locator (Grid Square)
Altitude (Mic-E)
E-mail message
Third-Party forwarding
Invalid Data/Test Data


\section{Comment Field}

In general, any APRS packet can contain a plain text comment (such as a
beacon message) in the Information field, immediately following the APRS
Data or APRS Data Extension.

There is no separator between the APRS data and the comment unless
otherwise stated.

The comment may contain any printable ASCII characters (except | and ~,
which are reserved for TNC channel switching).

The maximum length of the comment field depends on the report — details
are included in the description of each report.

In special cases, the Comment field can also contain further APRS data:

\begin{itemize}


\item Altitude in comment text (see Chapter 6: Time and Position Formats), or
in Mic-E status text (see Chapter 10: Mic-E Data Format).

\item Maidenhead Locator (grid square), in a Mic-E status text field (see
Chapter 10: Mic-E Data Format) or in a Status Report (see Chapter 16:
Status Reports).

\item Bearing and Number/Range/Quality parameters (/BRG/NRQ), in DF
reports (see Chapter 7: APRS Data Extensions).

\item Area Object Line Widths (see Chapter 11: Object and Item Reports).

\item Signpost Objects (see Chapter 11: Object and Item Reports).

\item Weather and Storm Data (see Chapter 12: Weather Reports).

\item Beam Heading and Power, in Status Reports (see Chapter 16: Status
Reports).

\end{itemize}

\section{Base-91 Notation}

Two APRS data formats use base-91 notation: lat/long coordinates in
compressed format (see Chapter 9) and the altitude in Mic-E format (see
Chapter 10).

Base-91 data is compressed into a short string of characters. All the
characters are printable ASCII, with character codes in the range 33–124
decimal (i.e. ! through |).

To compute the base-91 ASCII character string for a given data value, the
value is divided by progressively reducing powers of 91 until the remainder
is less than 91. At each step, 33 is added to the modulus of the division
process to obtain the corresponding ASCII character code.

For example, for a data value of 12345678:
\begin{verbatim}
12345678 / 913 = modulus 16, remainder 288542
288542 / 912 = modulus 34, remainder 6988
6988 / 911
= modulus 76, remainder 72
\end{verbatim}

The four ASCII character codes are thus 49 (i.e. 16+33), 67 (i.e. 34+33), 109
(i.e. 76+33) and 105 (i.e. 72+33), corresponding to the ASCII string 1Cmi.

\section{APRS Data Units}

For historical reasons there is some lack of consistency between units of data
in APRS packets — some speeds are in knots, others in miles per hour; some
altitudes are in feet, others in meters, and so on. It is emphasized that this
specification describes the units of data as they are transmitted on-air. It is
the responsibility of APRS applications to convert the on-air units to more
suitable units if required.

The default GPS earth datum is World Geodetic System (WGS) 1984.


