\chapter{Chapter 6: Time and Position Formats}


\section{Time Formats}

APRS timestamps are expressed in three different ways:

\begin{itemize}

\item Day/Hours/Minutes format

\item Hours/Minutes/Seconds format

\item Month/Day/Hours/Minutes format


\end{itemize}

In all three formats, the 24-hour clock is used.
Day/Hours/Minutes (DHM) format is a fixed 7-character field, consisting of
a 6-digit day/time group followed by a single time indicator character (z or
/). The day/time group consists of a two-digit day-of-the-month (01–31) and
a four-digit time in hours and minutes.
Times can be expressed in zulu (UTC/GMT) or local time. For example:
092345z
092345/

is 2345 hours zulu time on the 9th day of the month.
is 2345 hours local time on the 9th day of the month.

It is recommended that future APRS implementations only transmit zulu
format on the air.

Note: The time in Status Reports may only be in zulu format.

Hours/Minutes/Seconds (HMS) format is a fixed 7-character field,
consisting of a 6-digit time in hours, minutes and seconds, followed by the h
time-indicator character. For example:
234517h

is 23 hours 45 minutes and 17 seconds zulu.

Note: This format may not be used in Status Reports.

Month/Day/Hours/Minutes (MDHM) format is a fixed 8-character field,
consisting of the month (01–12) and day-of-the-month (01–31), followed by
the time in hours and minutes zulu. For example:
10092345

is 23 hours 45 minutes zulu on October 9th.

This format is only used in reports from stand-alone “positionless” weather
stations (i.e. reports that do not contain station position information).


\section{Use of Timestamps}

When a station transmits a report without a timestamp, an APRS receiving
station can make an internal record of the time it was received, if required.
This record is the receiving station’s notion of the time the report was
created.

On the other hand, when a station transmits a report with a timestamp, that
timestamp represents the transmitting station’s notion of the time the report
was created.

In other words, reports sent without a timestamp can be regarded as real-time,
“current” reports (and the receiving station has to record the time they were
received), whereas reports sent with a timestamp may or may not be realtime, and may possibly be (very) “old”.
Four APRS Data Type Identifiers specify whether or not a report contains a
timestamp, depending on whether the station has APRS messaging capability
or not:

No APRS
Messaging

With APRS
Messaging

(Current/real-time)

Report without timestamp

!

=

(Old/non-real-time)

Report with timestamp

/

@

Stations without APRS messaging capability are typically stand-alone
trackers or digipeaters. Stations reporting without a timestamp are generally
(but not necessarily) fixed stations.

\section{Latitude Format}

Latitude is expressed as a fixed 8-character field, in degrees and decimal
minutes (to two decimal places), followed by the letter N for north or S for
south.

Latitude degrees are in the range 00 to 90. Latitude minutes are expressed as
whole minutes and hundredths of a minute, separated by a decimal point.

For example:
4903.50N

is 49 degrees 3 minutes 30 seconds north.

In generic format examples, the latitude is shown as the 8-character string
ddmm.hhN (i.e. degrees, minutes and hundredths of a minute north).

\section{Longitude Format}

Longitude is expressed as a fixed 9-character field, in degrees and decimal
minutes (to two decimal places), followed by the letter E for east or W for
west.


Longitude degrees are in the range 000 to 180. Longitude minutes are
expressed as whole minutes and hundredths of a minute, separated by a
decimal point.

For example:
07201.75W

is 72 degrees 1 minute 45 seconds west.

In generic format examples, the longitude is shown as the 9-character string
dddmm.hhW (i.e. degrees, minutes and hundredths of a minute west).

\section{Position Coordinates}

Position coordinates are a combination of latitude and longitude, separated
by a display Symbol Table Identifier, and followed by a Symbol Code. For
example:

4903.50N/07201.75W-

The / character between latitude and longitude is the Symbol Table
Identifier (in this case indicating use of the Primary Symbol Table), and the –
character at the end is the Symbol Code from that table (in this case,
indicating a “house” icon).

A description of display symbols is included in Chapter 20: APRS Symbols.
The full Symbol Table listing is in Appendix 2.

\section{Position Ambiguity}

In some instances — for example, where the exact position is not known —
the sending station may wish to reduce the number of digits of precision in
the latitude and longitude. In this case, the mm and hh digits in the latitude
may be progressively replaced by a V (space) character as the amount of
imprecision increases. For example:
4903.5VN

represents latitude to nearest 1/10th of a minute.

4903.VVN

represents latitude to nearest minute.

490V.VVN

represents latitude to nearest 10 minutes.

49VV.VVN

represents latitude to nearest degree.

The level of ambiguity specified in the latitude will automatically apply to
the longitude as well — it is not necessary to include any V characters in the
longitude.
For example, the coordinates:
4903.VVN/07201.75W-

represent the position to the nearest minute. That is, the hundredths of
minutes of latitude and longitude may take any value in the range 00–99.




Thus the station may be located anywhere inside a bounding box having the
following corner coordinates:

\begin{itemize}

\item North West corner: 49 deg 3.99 mins N, 72 deg 1.99 mins W
\item North East corner: 49 deg 3.99 mins N, 72 deg 1.00 mins W
\item South East corner: 49 deg 3.00 mins N, 72 deg 1.00 mins W
\item South West corner: 49 deg 3.00 mins N, 72 deg 1.99 mins W

\end{itemize}


\section{Default Null Position}

Where a station does not have any specific position information to transmit
(for example, a Mic-E unit without a GPS receiver connected to it), the
station must transmit a default null position in the location field.

The null position corresponds to 0° 0' 0" north, 0° 0' 0" west.
The null position should be include the \. symbol (unknown/indeterminate
position).

For example, a Position Report for a station with unknown position
will contain the coordinates …0000.00N\00000.00W.…

\section{Maidenhead Locator (Grid Square)}

An alternative method of expressing a station’s location is to provide a
Maidenhead locator (grid square). There are four ways of doing this:

\begin{itemize}
  
\item In a Status Report — e.g. IO91SX/(/- represents the symbol for a “house”).

\item In Mic-E Status Text — e.g. IO91SX/G
(/G indicates a “grid square”).

\item In the Destination Address — e.g. IO91SX. (obsolete).

\item In AX.25 beacon text, with the [ APRS Data Type Identifier — e.g.
[IO91SX] (obsolete).

\end{itemize}

Grid squares may be in 6-character form (as above) or in the shortened
4-character form (e.g. IO91).

\section{NMEA Data}

APRS recognizes raw ASCII data strings conforming to the NMEA 0183
Version 2.0 specification, originating from navigation equipment such as
GPS and LORAN receivers. It is recommended that APRS stations interpret
at least the following NMEA Received Sentence types:

\begin{itemize}

\item GGA Global Positioning System Fix Data
\item GLL Geographic Position, Latitude/Longitude Data
\item RMC Recommended Minimum Specific GPS/Transit Data
\item VTG Velocity and Track Data
\item WPT Way Point Location
% footnote WPT as typo, as mentioned in 1.1


\end{itemize}



\section{Altitude}

Altitude may be expressed in two ways:

\begin{itemize}
\item In the comment text.
\item In Mic-E format.
\end{itemize}

\textbf{Altitude in Comment Text --} The comment may contain an altitude value,
in the form /A=aaaaaa, where aaaaaa is the altitude in feet. For example:
/A=001234. The altitude may appear anywhere in the comment.

\textbf{Altitude in Mic-E format --} The optional Mic-E status field can contain
altitude data. See Chapter 10: Mic-E Data Format.

