
\chapter{Chapter 7: APRS Data Extensions}


A fixed-length 7-byte field may follow APRS position data. This field is an
APRS Data Extension. The extension may be one of the following:

\section{Course and Speed}

\begin{tabular}{p{0.25\linewidth}  p{0.6\linewidth}}

 CSE/SPD & Course and Speed (this may be followed by a further 8 bytes
 containing DF bearing and Number/Range/Quality parameters) \\
  & \\
 DIR/SPD & Wind Direction and Wind Speed \\
  & \\
 PHGphgd & Station Power and Effective Antenna Height/Gain/
Directivity \\
  & \\
 RNGrrrr & Pre-Calculated Radio Range \\
  & \\
 DFSshgd & DF Signal Strength and Effective Antenna Height/Gain \\
  & \\
 Tyy/Cxx & Area Object Descriptor \\
 & \\
 & \\
\end{tabular}


The 7-byte CSE/SPD Data Extension can be used to represent the course and
speed of a vehicle or APRS Object.
The course is expressed in degrees (001-360), clockwise from due north. The
speed is expressed in knots. A slash / character separates the two.

For example:
088/036 represents a course 88 degrees, traveling at 36 knots.

If the course and speed are unknown or not relevant, they can be set
to 000/000 or
.../... or \textvisiblespace \textvisiblespace \textvisiblespace
/\textvisiblespace \textvisiblespace \textvisiblespace .

Note: In the special case of DF reports, a course of 000 means that the DF
station is fixed. If the course is non-zero, the station is mobile.

\section {Wind Direction and Wind Speed}

The 7-byte DIR/SPD Data Extension can be used to represent the wind
direction and sustained one-minute wind speed in a Weather Report.
The wind direction is expressed in degrees (001-360), clockwise from due
north. The speed is expressed in knots. A slash / character separates the two.

For example:

220/004 represents a wind direction of 220 degrees and a speed of 4 knots.

If the wind direction and speed are unknown or not relevant, they can be set
to 000/000 or .../... or  \textvisiblespace \textvisiblespace \textvisiblespace
/\textvisiblespace \textvisiblespace \textvisiblespace .

\section{Power, Effective Antenna Height/Gain/Directivity}

The 7-byte PHGphgd Data Extension specifies the transmitter power,
effective antenna height-above-average-terrain, antenna gain and antenna
directivity. APRS uses this information to plot radio range circles around
stations.


The 7 characters of this Data Extension are encoded as follows:

\begin{tabular}{lcr}

Characters 1–3: & PHG & (fixed) \\

Character 4: & p & Power code \\

Character 5: & h & Height code \\

Character 6: & g & Antenna gain code \\

Character 7: & d & Directivity code \\

& \\
& \\
\end{tabular}



% Start Chart


The PHG codes are listed in the table below:
\begin{table}
  \caption {PHG Codes}
  \begin{tabular}{|c|c|c|c|c|c|c|c|c|c|c|c|}
    \hline      
    phgd Code: & 0 & 1 & 2 & 3 & 4 & 5 & 6 & 7 & 8 & 9 & Units \\
    \hline  
    Power & 0 & 1 & 4 & 9 & 16 & 25 & 36 & 49 & 64 & 81 & watts \\
    \hline  
    Height & 10 & 20 & 40 & 80 & 160 & 320 & 640 & 1280 & 2560 & 5120 & feet \\
    \hline  
    Gain & 0 & 1 & 2 & 3 & 4 & 5 & 6 & 7 & 8 & 9 & dB \\
    \hline  
    Directivity & omni & 45 NE & 90 E & 135 SE & 180 S & 225 SW & 270 W & 315 NW & 360 N & & deg \\
    \hline  
  \end{tabular}
\end{table}


% End Chart

The height code represents the effective height of the antenna above average
local terrain, not above ground or sea level — this is to provide a rough
indication of the antenna’s effectiveness in the local area.

The height code may in fact be any ASCII character 0–9 and above. This is
so that larger heights for balloons, aircraft or satellites may be specified.

For example:
\begin{itemize}
  
\item : is the height code for 10240 feet (approximately 1.9 miles).
\item ; is the height code for 20480 feet (approximately 3.9 miles), and so on.
\end{itemize}

The Directivity code offsets the PHG circle by one third in the indicated
direction. This means a front-to-back ratio of 2 to 1. Most often this is used
to indicate a favored direction or a null, even if an omni antenna is at the site.

An example of the PHG Data Extension:

\begin{itemize}
\item PHG5132
\end{itemize}

\begin{itemize}
% means

\item a power of 25 watts,
\item an antenna height of 20 feet above the average local terrain,
\item an antenna gain of 3 dB,
\item and maximum gain due east.
\end{itemize}


\section{Range Circle Plot}

On receipt, APRS uses the p, h, g and d codes to calculate the usable radio
range (in miles), for plotting a range circle representing the local radio
horizon around the station. The radio range is calculated as follows:
power = p2

Height-above-average-terrain (haat) = 10 x 2h
gain = 10(g/10)
range = –( 2 x haat x –( (power/10) x (gain/2) ) )

Thus, for PHG5132:
power = 52 = 25 watts
haat = 10 x 21 = 20 feet
gain = 10(3/10) = 1.995262
range = –( 2 x 20 x –( (25/10) x (1.995262/2) ) )

~ 7.9 miles
As the direction of maximum gain is due east, APRS will draw a range circle
of radius 8 miles around the station, offset by 2.7 miles (i.e. one third of 8
miles) in an easterly direction.
Note: In the absence of any PHG data, stations are assumed to be running 10
watts to a 3dB omni antenna at 20 feet, resulting in a 6-mile radius range
circle, centered on the station.

\section {Pre-Calculated Radio Range}

The 7-byte RNGrrrr Data Extension allows users to transmit a precalculated omni-directional radio range, where rrrr is the range in miles
(with leading zeros).
For example, RNG0050 indicates a radio range of 50 miles.
APRS can use this value to plot a range circle around the station.

\section{Omni-DF Signal Strength}

The 7-byte DFSshgd Data Extension lets APRS localize jammers by plotting
the overlapping signal strength contours of all stations hearing the signal.
This Omni-DF format replaces the PHG format to indicate DF signal
strength, in that the transmitter power field is replaced with the relative
signal strength (s) from 0 to 9.


\subsection{DFS Codes}

\begin{table}
  \caption {shgd Code:}
  \begin{tabular}{|c|c|c|c|c|c|c|c|c|c|c|c|}
    \hline      
    shgd Code: & 0 & 1 & 2 & 3 & 4 & 5 & 6 & 7 & 8 & 9 & Units \\
    \hline  
    Strength & 0 & 1 & 2 & 3 & 4 & 5 & 6 & 7 & 8 & 9 & S-points \\
    \hline  
    Height & 10 & 20 & 40 & 80 & 160 & 320 & 640 & 1280 & 2560 & 5120 & feet \\
    \hline  
    Gain & 0 & 1 & 2 & 3 & 4 & 5 & 6 & 7 & 8 & 9 & dB \\
    \hline  
    Directivity & omni & 45 NE & 90 E & 135 SE & 180 S & 225 SW & 270 W & 315 NW & 360 N & & deg \\
    \hline  
  \end{tabular}
\end{table}


For example, DFS2360 represents a weak signal (around strength S2) heard
on an omni antenna with 6 dB gain at 80 feet.

A signal strength of zero (0) is particularly significant, because APRS uses
these 0 signal reports to draw (usually black) circles where the jammer is not
heard. These black circles are extremely valuable since there will be a lot
more reports from stations that do not hear the jammer than from those that
do. This quickly eliminates a lot of territory.

\section {Bearing and Number/Range/Quality}

DF reports contain an 8-byte field /BRG/NRQ that follows the CSE/SPD Data
Extension, specifying the course, speed, bearing and NRQ (Number/Range/
Quality) value of the report. NRQ indicates the Number of hits, the
approximate Range and the Quality of the report.

For example, in:
…088/036/270/729…

course = 88 degrees, speed = 36 knots,
bearing = 270 degrees, N = 7, R = 2, Q = 9

If N is 0, then the NRQ value is meaningless. Values of N from 1 to 8 give an
indication of the number of hits per period relative to the length of the time
period — thus a value of 8 means 100\% of all samples possible got a hit. A
value of 9 for N indicates to other users that the report is manual.
The N value is not processed, but is just another indicator from the automatic
DF units.

The range limits the length of the line to the original map’s scale of the
sending station. The range is 2R so, for R=4, the range will be 16 miles.
Q is a single digit in the range 0–9, and provides an indication of bearing
accuracy:

\begin{tabular}{|c|c|c |c|c|}
\hline
Q & Bearing Accuracy & \hspace{2em} & Q & Bearing Accuracy \\
\hline
0 & Useless & & 5 & < 16 deg \\
\hline
1 & < 240 deg & & 6 & < 8 deg \\
\hline
2 & < 120 deg & & 7 & < 4 deg \\
\hline
3 & < 64 deg & & 8 & < 2 deg \\
\hline
4 & < 32 deg & & 9 & < 1 deg (best) \\
\hline
\end{tabular}



If the course and speed parameters are not appropriate, they should have the
value 000/000 or .../... or \textvisiblespace  \textvisiblespace  \textvisiblespace /\textvisiblespace  \textvisiblespace  \textvisiblespace

\section {Area Object Descriptor}

The 7-byte \TypeWriterFont{Tyy/Cxx} Data Extension is an Area Object Descriptor. The T
parameter specifies the type of object (square, circle, triangle, etc) and the
/C parameter specifies its fill color.
Area Objects are described in Chapter 11: Object and Item Reports.

