\chapter{Chapter 8: Position and DF Report Data Formats}

\section{Position Reports}

Lat/Long Position Reports are contained in the Information field of an APRS AX.25 frame.

The following diagrams show the permissible formats of these reports,
together with some examples. The gray areas indicate optional fields, and the
shaded (yellow) characters are literal ASCII characters. In all cases there is a
maximum of 43 characters after the Symbol Code.

% \subsection*{Lat/Long Position Report Format — without Timestamp}
% Lat/Long Position Report Format — without Timestamp
% \subsection{``Basic Position Reports''} % Basic (not DF) 

\begin{table}[H]
  \begin{tabular}{r|c|c|c|c|c|c|}
    \hline
    \multicolumn {6}{r}{Lat/Long Position Report Format — without Timestamp} \\
    \hline
    & ! or & Lat & Sym Table ID & Long & Symbol Code & Comment (max 43 chars) \\
    & =    &     &              &      &             &                        \\
    \hline
    Bytes: & 1 & 8 & 1 & 9 & 1 & 0-43 \\
    \hline
  \end{tabular}
  
  \vspace{2em}

  \begin{tabular}{l}
    \multicolumn {1}{l}{Examples} \\
    \hline
    !4903.50N/07201.75W-Test \hspace{1ex} 001234 \hspace{1ex} (no timestamp, no APRS messaging, with comment.) \\
    !4903.50N/07201.75W-Test \hspace{1ex} /A=001234 \hspace{1ex} (no timestamp, no APRS messaging, altitude = 1234 ft.) \\
    !49\textvisiblespace\textvisiblespace.N/072\textvisiblespace\textvisiblespace.\textvisiblespace\textvisiblespace W- \hspace{1ex}  (no timestamp, no APRS messaging, location to nearest degree.) \\
    TheNet\textvisiblespace X-1J4\textvisiblespace\textvisiblespace(BFLD)!4903.50N/07201.75Wn  (no timestamp, no APRS messaging...,) \\
    \hspace{25ex}  (... with X1J node header string.) \\
    \hline
    \hline
  \end{tabular}
  \caption{Lat/Long Position Report Format — without Timestamp}
\end{table}

% \subsection*{Lat/Long Position Report Format — with Timestamp}
% Lat/Long Position Report Format — with Timestamp
             
\begin{table}[H]

  \begin{tabular}{r|c|c|c|c|c|c|c|}
    \hline
    \multicolumn {8}{l}{Lat/Long Position Report Format — with Timestamp} \\
    \hline
    & / or  & Time & Lat & Sym Table ID & Long & Symbol Code & Comment (max 43 chars) \\
    & @     & DHM / &     &              &      &             &  \\
    &       &  HMS  &     &              &      &             &  \\
    \hline
    Bytes & 1 & 7 & 8 & 1 & 9 & 1 & 0-43 \\
    \hline
  \end{tabular}
  
  \vspace{2em}
  
  \begin{tabular}{ll}
    \multicolumn {2}{l}{Examples} \\
    \hline
    /092345z4903.50N/07201.75W>Test1234 & with timestamp  \\
    & no APRS messaging, zulu time, with comment. \\
    @092345/4903.50N/07201.75W>Test1234 &  with timestamp, \\
    & with APRS messaging, local time, with comment. \\
    \hline
    \hline
  \end{tabular}
  
  \caption{Lat/Long Position Report Format — with Timestamp}
\end{table}

% \subsection*{Lat/Long Position Report Format — with Data Extension (no Timestamp)}
% Lat/Long Position Report Format — with Data Extension (no Timestamp)

\begin{table}[H]
  
  \begin{tabular}{r|c|c|c|c|c|c|c|}
    \hline
    \multicolumn{8}{l}{Lat/Long Position Report Format — with Data Extension (no Timestamp)} \\
    \hline 
    & ! or & Lat & Sym    & Long & Symbol & Data Ext $\dagger$ & Comment (max 36 chars) \\
    & =    &     & Table  &      & Code   &       & \\
    &      &     &        &      &        &       & \\
    \hline
    Bytes: & 1 & 8 & 1 & 9 & 1 & 7 & 0-36  \\
    \hline
        
  \end{tabular}
  \vspace{2em}
  
  Where $\dagger$ is one of:
  \begin{itemize}
  \item Course/Speed
  \item Power/Height/Gain/Dir
  \item Radio Range
  \item DF Signal Strength
  \end{itemize}

  \vspace {2em}
  
  \underline{Examples}
  \vspace{1em}
  
  \begin{itemize}
  \item[] =4903.50N/07201.75W\#PHG5132 \hspace{1ex} no timestamp, with APRS messaging, with PHG. 
  \item[] =4903.50N/07201.75W\_225/000g000t050r000p001...h00b10138dU2k \hspace{1ex} weather report.
  \end{itemize}

  \doublerule
  \caption{Lat/Long Position Report Format — with Data Extension (no Timestamp)}
\end{table}

% \subsection*{Lat/Long Position Report Format — with Data Extension and Timestamp}
% Lat/Long Position Report Format — with Data Extension and Timestamp

\begin{table}[H]
  \begin{tabular}{r|c|c|c|c|c|c|c|c|}
    \hline
    \multicolumn{8}{l}{Lat/Long Position Report Format — with Data Extension and Timestamp} \\
    \hline 
    & / or & Time    &     & Sym    & Long & Symbol & Data Ext $\dagger$ & Comment (max 36 chars) \\
    & @    & DHM /   & Lat & Table  &      & Code   &       & \\
    &      & HMS     &     &        &      &        &       & \\
    &      &         &     &        &      &        &       & \\
    &      &         &     &        &      &        &       & \\
    \hline
    Bytes: & 1 & 7 & 8 & 1 & 9 & 1 & 7 & 0-36  \\
    \hline
        
  \end{tabular}
  \vspace{2em}
  
  Where $\dagger$ is one of:
  \begin{itemize}
  \item Course/Speed
  \item Power/Height/Gain/Dir
  \item Radio Range
  \item DF Signal Strength
  \end{itemize}

  \vspace {2em}
  
  \underline{Examples}
  \vspace{1em}
  
  \begin{itemize}
  \item[] @092345/4903.50N/07201.75W>088/036
  \item (with timestamp, with APRS messaging, local time,course/speed.)
  \item[]
  \item[] @234517h4903.50N/07201.75W>PHG5132
  \item (with timestamp, APRS messaging, hours/mins/secs time, PHG.)
  \item[]
  \item[] @092345z4903.50N/07201.75W>RNG0050
  \item (with timestamp, APRS messaging, zulu time, radio range.)
  \item[]
  \item[] /234517h4903.50N/07201.75W>DFS2360
  \item (with timestamp, hours/mins/secs time, DF,no APRS messaging.)
  \item[]
  \item[] @092345z4903.50N/07201.75W\_090/000g000t066r000p000...dUII
  \item[] (weather report.)
  \end{itemize}

  \doublerule
  
  \caption{Lat/Long Position Report Format — with Data Extension and Timestamp}
\end{table}


% Maidenhead Locator Beacon
\begin{table}[H]

  \begin{tabular}{l|c|c|c|c|}
  \hline
  \multicolumn{5}{c}{Maidenhead Locator Beacon} \\
  \hline
  & [ & Grid Locator & ] & Comment \\
  Bytes & 1 & 4 or 6 & 1 & n \\
  \hline
  \noalign{\vskip 2em}
  \multicolumn{5}{l}{\underline{Examples}} \\
  \multicolumn{5}{l}{[IO91SX] 35 miles NNW of London} \\
  \multicolumn{5}{l}{[IO91]} \\
  \end{tabular}
  
  \doublerule
  \caption{Maidenhead Locator Beacon}
\end{table}


% Raw NMEA Position Report Format

\begin{table}[H]
  \begin{tabular}{|l|c|c|}
    \hline
    \multicolumn{3}{|c|}{Raw NMEA Position Report Format} \\
    \hline
    \multicolumn{3}{|c|}{NMEA Received Sentence} \\
    \hline
    & \$ & ...,...,...,...,...,...,... \\
    \hline
    Bytes & 1 & 25-209 \\
    \hline 
    % Raw NMEA Position Report Format
  \end{tabular}
  \vspace{2em}
  
  \underline{Examples}
  \vspace{1.5em}
  
  \begin{verbatim}
  $GPGGA,102705,5157.9762,N,00029.3256,W,1,04,2.0,75.7,M,47.6,M,,*62
  $GPGLL,2554.459,N,08020.187,W,154027.281,A
  $GPRMC,063909,A,3349.4302,N,11700.3721,W,43.022,89.3,291099,13.6,E*52
  $GPVTG,318.7,T,,M,35.1,N,65.0,K*69
  \end{verbatim}

  \doublerule
  \caption{Raw NMEA Position Report Format}
  \end{table}
  
\section{DF Reports}

\subsection*{DF Reports (Direction Finding)\footnote {Added by Ed.}} % DF

DF Reports are contained in the Information field of an APRS AX.25
frame.  The Bearing and Number/Range/Quality (BRG/NRQ) parameters
follow the Data Extension field.  Note: The BRG/NRQ parameters are
only meaningful when the report contains the DF symbol (i.e. the
Symbol Table ID is / and the Symbol Code is $\backslash$).

 Note: If the DF station is fixed, the Course value is zero. If the station is
moving, the Course value is non-zero.


% DF Report Format without Timestamp

\begin{table}[H]
  
  \begin{tabular}{r|c|c|c|c|c|c|c|c|}
    \hline
    \multicolumn{9}{l}{DF Report Format — without Timestamp} \\
    \hline 
    & ! or & Lat & Sym    & Long & Symbol & Data Ext $\dagger$ & /BRG/NRQ  & Comment (max 28 chars) \\
    & =    &     & Table  &      & Code   &       &   & \\
    &      &     & ID     &      &        &       &   & \\
    &      &     &  /     &      &   $\backslash$    &       &   & \\
    \hline
    Bytes: & 1 & 8 & 1 & 9 & 1 & 7 & 8 & 0-28  \\
    \hline
        
  \end{tabular}
  \vspace{2em}
  
  Where $\dagger$ is one of:
  \begin{itemize}
  \item Course/Speed
  \item Power/Height/Gain/Dir
  \item Radio Range
  \item DF Signal Strength
  \end{itemize}

  \vspace {2em}
  
  \underline{Examples}
  \vspace{1em}
  
\begin{verbatim}
=4903.50N/07201.75W\088/036/270/729
=4903.50N/07201.75W\000/036/270/729
\end{verbatim}
\vspace{2em}

\begin{itemize}
\item First example is: no timestamp, course/speed/
bearing/NRQ, with APRS messaging.
DF station moving (CSE is non-zero).

\item Second example is: Same report, DF station fixed
(CSE=000).
\end{itemize}

  \doublerule
  \caption{DF Report Format -- without Timestamp}
\end{table}




  % DF Report Format — with Timestamp

\begin{table}[H]
  
  \begin{tabular}{r|c|c|c|c|c|c|c|c|c|}
    \hline
    \multicolumn{10}{l}{DF Report Format — with Timestamp} \\
    \hline 
    & / or & Time  &     & Sym    & Long & Symbol & Data Ext $\dagger$ & /BRG/NRQ  & Comment (max 28 chars) \\
    & @    & DHM/  & Lat & Table  &      & Code   &       &  & \\
    &      &  HMS  &     &   ID   &      &   $\backslash$    &       &  & \\
    &      &       &     &   /    &      &        &       & & \\
    
    \hline
    Bytes: & 1 & 7 & 8 & 1 & 9 & 1 & 7 & 8 & 0-28  \\
    \hline
        
  \end{tabular}
  \vspace{2em}
  
  Where $\dagger$ is one of:
  \begin{itemize}
  \item Course/Speed
  \item Power/Height/Gain/Dir
  \item Radio Range
  \item DF Signal Strength
  \end{itemize}

  \vspace {2em}
  
  \underline{Examples}
  \vspace{1em}
  
\begin{verbatim}
@092345z4903.50N/07201.75W\088/036/270/729
/092345z4903.50N/07201.75W\000/000/270/729
\end{verbatim}
\vspace {2em}

\begin{itemize}
\item First example is:with timestamp, course/speed/
bearing/NRQ, with APRS messaging.

\item Second example is: with timestamp, bearing/NRQ, no
course/speed, no APRS messaging.
\end{itemize}

  \doublerule
  \caption{DF Report Format with Timestamp}
\end{table}
