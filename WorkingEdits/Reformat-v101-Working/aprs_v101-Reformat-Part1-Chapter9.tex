% This chapter is a mess--ehb

\chapter{Chapter 9: Compressed Position Report Data Formats}


In compressed data format, the Information field contains the station’s
latitude and longitude, together with course and speed or pre-calculated radio
range or altitude.
This information is compressed to minimize the length of the transmitted
packet (and therefore improve its chances of being received correctly under
less than ideal conditions).
The Information field also contains a display Symbol Code, and there may
optionally be a plain text comment (uncompressed) as well.

The Advantages of
Data Compression

Compressed data format may be used in place of the numeric lat/long
coordinates already described, such as in the !, /, @ and = formats.
Data compression has several important benefits:

\begin{itemize}
  
\item Fully backwards compatible with all existing formats.

\item Fully supports any comment string.

\item Speed is accurate to +/-1 mph up to about 40 mph and within 3\%
  at 600 mph.

\item Altitude in feet is accurate to +/- 0.4\% from 1 foot to 3000 miles.

\item Consistent one-algorithm processing of compressed latitude and
  longitude.

\item Improved position to 1 foot worldwide.

\item Pre-calculated radio range, compressed to one byte.

\item Potential 50\% compression of every position format on the air.

\item Potential 40\% reduction of raw GPS NMEA data length.

\item Additional 7-byte reduction for NEMA GGA altitudes.

\item Support for TNC compression at the NMEA source (from the GPS
  receiver).

\item Digipeater compression of old NMEA trackers on the fly.

\item Usage is optional in all cases.

\end{itemize}

The only minor disadvantages are that the course only resolves to +/- 2
degrees, and this format does not support PHG.



Compressed Data
Format

Compressed data may be generated in several ways:

\begin{itemize}
    
\item by APRS software.

\item pre-entered manually into a digipeater’s beacon text.

\item by a digipeater converting raw tracker NMEA packets to compressed.

\end{itemize}


[In future, there is the possibility that a Kantronics KPC-3 or other tracker
TNC will be able to compress data directly from an attached GPS receiver].
In all cases the compressed format is a fixed 13-character field:
/YYYYXXXX$csT

where /

is the Symbol Table Identifier
is the compressed latitude
is the compressed longitude
is the Symbol Code
is the compressed course/speed or
compressed pre-calculated radio range or
compressed altitude
is the compression type indicator

YYYY
XXXX
$
cs

T

Compressed Position Data
Sym
Table
ID

Compressed
Lat

Compressed
Long

YYYY

XXXX

Symbol
Code

Compressed
Course/Speed
Compressed Radio
Range

Comp
Type

T

Compressed
Altitude
Bytes:

1

4

4

1

2

1

Compressed format can be used in place of lat/long position format anywhere
that …ddmm.hhN/dddmm.hhW$xxxxxxx… occurs.
All bytes except for the / and $ are base-91 printable ASCII characters
(!..{). These are converted to numeric values by subtracting 33 from the
decimal ASCII character code. For example, # has an ASCII code of 35, and
represents a numeric value of 2 (i.e. 35-33).

Symbol

The presence of the leading Symbol Table Identifier instead of a digit
indicates that this is a compressed Position Report and not a normal lat/long
report.


Lat/Long Encoding

The values of YYYY and XXXX are computed as follows:
YYYY

is 380926 x (90 – latitude) [base 91]
latitude is positive for north, negative for south, in degrees.

XXXX

is 190463 x (180 + longitude) [base 91]
longitude is positive for east, negative for west, in degrees.

For example, for a longitude of 72° 45' 00" west (i.e. -72.75 degrees), the
math is 190463 x (180 – 72.75) = 20427156. Because this is to base 91, it is
then necessary to progressively divide this value by reducing powers of 91,
to obtain the numerical values of X:
20427156 / 913
80739 / 912
6210 / 911

= 27, remainder 80739
= 9, remainder 6210
= 68, remainder 22

To obtain the corresponding ASCII characters, 33 is added to each of these
values, yielding 60 (i.e. 27+33), 42, 101 and 55. From the ASCII Code Table
(in Appendix 3), this corresponds to <*e7 for XXXX.

Lat/Long Decoding

To decode a compressed lat/long, the reverse process is needed. That is, if
YYYY is represented as y1y2y3y4 and XXXX as x1x2x3x4, then:
3

2

Lat = 90 - ((y1-33) x 91 + (y2-33) x 91 + (y3-33) x 91 + y4-33) / 380926
3

2

Long = -180 + ((x1-33) x 91 + (x2-33) x 91 + (x3-33) x 91 + x4-33) / 190463

For example, if the compressed value of the longitude is <*e7 (as computed
above), the calculation becomes:
3

2

Long = -180 + (27 x 91 + 9 x 91 + 68 x 91 + 22) / 190463

= -180 + (20346417 + 74529 + 6188 + 22) / 190463
= -180 + 107.25
= -72.75 degrees

Course/Speed,
Pre-Calculated
Radio Range and
Altitude

The two cs bytes following the Symbol Code character can contain either
the compressed course and speed or the compressed pre-calculated radio
range or the station’s altitude. These two bytes are in base 91 format.
In the special case of c = V (space), there is no course, speed or range
data, in which case the csT bytes are ignored.
Course/Speed — If the ASCII code for c is in the range ! to z inclusive —
corresponding to numeric values in the range 0–89 decimal (i.e. after
subtracting 33 from the ASCII code) — then cs represents a compressed
course/speed value:




course = c x 4
speed = 1.08s – 1

For example, if the cs characters are 7P, the corresponding values of c and
s (after subtracting 33 from the ASCII character code) are 22 and 47
respectively. Substituting these values in the above equations:
course = 22 x 4 = 88 degrees
speed = 1.0847 – 1 = 36.2 knots

Pre-Calculated Radio Range — If c = {, then cs represents a compressed
pre-calculated radio range value:
range = 2 x 1.08s

For example, if the cs bytes are {?, the ASCII code for ? is 63, so the value
of s is 30 (i.e. 63-33). Thus:
range = 2 x 1.0830

~ 20 miles
So APRS will draw a circle of radius 20 miles around the station plot on the
screen.

The Compression
Type (T) Byte

The T byte follows the cs bytes. The T byte contains several bit fields
showing the GPS fix status, the NMEA source of the position data and the
origin of the compression.
The T byte is not meaningful if the c byte is V (space).
Compression Type (T) Byte Format

Bit:

Value:

7

6

5

Not used

Not used

GPS Fix

0

0

0 = old (last)
1 = current

4

3

NMEA Source
0 0 = other
0 1 = GLL
1 0 = GGA
1 1 = RMC

2

1

0

Compression Origin
0 0 0 = Compressed
0 0 1 = TNC BText
0 1 0 = Software (DOS/Mac/Win/+SA)
0 1 1 = [tbd]
1 0 0 = KPC3
1 0 1 = Pico
1 1 0 = Other tracker [tbd]
1 1 1 = Digipeater conversion

For example, if the compressed position was derived from an RMC sentence,
the fix is current, and the compression was performed by APRSdos software,
then the value of T in binary is 0 0 1 11 010, which equates to 58 decimal.
Adding 33 to this value gives the ASCII code for the T byte (i.e. 91), which
corresponds to the [ character.
Thus, using data from all the earlier examples, if the RMC sentence contains
(among other parameters) the following data:
Latitude
Longitude
Speed
Course
and:

=
=
=
=

49° 30' 00" north
72° 45' 00" west
36.2 knots
88°

the fix is current
compression is performed by APRSdos software
the display symbol is a “car”

then the complete 13-character compressed location field is transmitted as:

Altitude

/ YYYY

XXXX

$

csT

/ 5L!!

<*e7

>

7P[

If the T byte indicates that the raw data originates from a GGA sentence (i.e.
bits 4 and 3 of the T byte are 10), then the sentence contains an altitude
value, in feet. After compression, the compressed altitude data is placed in
the cs bytes, such that:
altitude = 1.002cs feet

For example, if the received cs bytes are S], the computation is as follows:

New Trackers

•

Subtract 33 from the ASCII code for each character:
c = 83 – 33 = 50
s = 93 – 33 = 60

•

Multiply c by 91 and add s to obtain cs:
cs = 50 x 91 + 60
= 4610

•

Then altitude = 1.0024610
= 10004 feet

Tracker firmware may compress GPS data directly to APRS compressed
format. They would use the ! Data Type Indicator, showing that the position
is real-time and that the tracker is not APRS-capable.
If the Position Report is not real-time, then the / Data Type Indicator can be
used instead, so that the latest fix time may be included.


Some digipeaters have the ability to convert raw NMEA strings from existing
trackers to compressed data format for further forwarding.

Old Trackers

These digipeaters will compress the data if the tracker Destination Address is
GPS. (Note: This is the 3-letter address GPS, not GPS*).
Trackers desiring for their packets to not be modified by the APRS network
will use any other valid generic APRS Destination Address.

Compressed data is contained in the AX.25 Information field, in these
formats:

Compressed Report
Formats

Compressed Lat/Long Position Report Format — no Timestamp
! or
=

Sym
Table
ID

Comp
Lat

Comp
Long

YYYY

XXXX

Symbol
Code

Compressed
Course/Speed

T

Comment
(max 40
chars)

1

0-40

Comp
Type

Compressed Radio
Range
Compressed
Altitude

Bytes:

1

1

4

4

1

Examples
=/5L!!<*e7>V sTComment

2

with APRS messaging. Note the V space character following the >
Symbol Code, indicating that there is no course/speed, radio range or
altitude. The sT characters are fillers and have no significance here.
with APRS messaging, RMC sentence, with course/speed.
with APRS messaging, with radio range.
with APRS messaging, GGA sentence, altitude.

=/5L!!<*e7>7P[
=/5L!!<*e7>{?!
=/5L!!<*e7OS]S

Compressed Lat/Long Position Report Format — with Timestamp

/ or

@

Time
DHM /
HMS

Sym
Table
ID

Comp
Lat

Comp
Long

YYYY

XXXX

Symbol
Code

Compressed
Course/Speed
Compressed Radio
Range

T

Comment
(max 40
chars)

1

0-40

Comp
Type

Compressed
Altitude
Bytes:

1

7

1

4

Example
@092345z/5L!!<*e7>{?!

4

1

2

with APRS messaging, timestamp, radio range.





\chapter{Chapter 10: Mic-E Data Format}


\section{Mic-E Data Format}

In Mic-E data format, the station’s position, course, speed and display
symbol, together with an APRS digipeater path and Mic-E Message Code,
are packed into the AX.25 Destination Address and Information fields.
The Information field can also optionally contain either Mic-E telemetry data
or Mic-E status. The Mic-E Status can contain the station’s Maidenhead
locator and altitude.

Mic-E packets can be very short. At the minimum, with no callsigns in the
Digipeater Addresses field and no optional telemetry data or Mic-E status
text, a complete Mic-E packet is just 25 bytes long (excluding FCS and
flags).

Mic-E data format is not only used in the Microphone Encoder unit; it is also
used in the PIC Encoder and in the Kenwood TH-D7 and TM-D700 radios.

\section{Mic-E Data Payload}

The Mic-E data format allows a large amount of data to be carried in a very
short packet. The data is split between the Destination Address field and the
Information field of a standard AX.25 UI-frame.

\subsection{Destination Address Field--} The 7-byte Destination Address field contains
the following encoded information:

\begin {itemize}
\item The 6 latitude digits.

\item A 3-bit Mic-E message identifier, specifying one of 7 Standard Mic-E
Message Codes or one of 7 Custom Message Codes or an Emergency
Message Code.

\item The North/South and West/East Indicators.

\item The Longitude Offset Indicator.

\item The generic APRS digipeater path code.
\end{itemize}

Although the destination address appears to be quite unconventional, it is
still a valid AX.25 address, consisting only of printable 7-bit ASCII values
(shifted one bit left) — see the Amateur Packet-Radio Link-Layer Protocol
specification for full details of the format of standard AX.25 addresses.
Information Field — This field contains the following data:


\begin{itemize}
  
\item The encoded longitude.

\item The encoded course and speed.

\item The display Symbol Code and Symbol Table Identifier.

\item An optional field, containing either Mic-E telemetry data or a Mic-E
status text string. The status string can contain plain text, Maidenhead
locator or the station’s altitude.

\end{itemize}

\subsection{Mic-E Destination}
\paragraph {Address Field}

The standard AX.25 Destination Address field consists of 7 bytes, containing
6 callsign characters and the SSID (plus a number of other bits that are not of
interest here). When used to carry Mic-E data, however, this field has a quite
different format:

\paragraph {Mic-E Data — DESTINATION ADDRESS FIELD Format}

Bytes:

Lat Digit 1
+ Message
Bit A

Lat Digit 2
+ Message
Bit B

Lat Digit 3
+ Message
Bit C

Lat Digit 4
+ N/S Lat
Indicator

Lat Digit 5
+ Longitude
Offset

Lat Digit 6
+ W/E Long
Indicator

APRS
Digi Path
Code

1

1

1

1

1

1

1

The Destination Address field contains:


  
Destination
Address Field
Encoding

\begin{itemize}

\item Six encoded latitude digits specifying degrees (digits 1 and 2), minutes
(digits 3 and 4) and hundredths of minutes (digits 5 and 6).

\item 3-bit Mic-E message identifier (message bits A, B and C).

\item North/South latitude indicator.

\item Longitude offset (adds 0 degrees or 100 degrees to the longitude
computation in the Information field).

\item West/East longitude indicator.

\item Generic APRS digipeater path (encoded in the SSID).

\end{itemize}
  
The table on the next page shows the encoding of the first 6 bytes of the
Destination Address field, for all combinations of latitude digit, the 3-bit
Mic-E message identifier (A/B/C), the latitude/longitude indicators and the
longitude offset.
The encoding supports position ambiguity.
The ASCII characters shown in the table are left-shifted one bit position prior
to transmission.



Mic-E Destination Address Field Encoding (Bytes 1–6)
Byte:

1-6

1-3

4

5

6

ASCII
Char

Lat
Digit

Message
A/B/C

N/S

Long
Offset

W/E

0

0

0

South

+0

1

1

0

South

2

2

0

3

3

0

4

4

5
6
7

Byte:

1-6

1-3

4

5

6

ASCII
Char

Lat
Digit

Message
A/B/C

N/S

Long
Offset

W/E

East

H

7

1 (Custom)

+0

East

I

8

1 (Custom)

South

+0

East

J

9

1 (Custom)

South

+0

East

K

space

1 (Custom)

0

South

+0

East

L

space

0

South

+0

East

5

0

South

+0

East

P

0

1 (Std)

North

+100

West

6

0

South

+0

East

Q

1

1 (Std)

North

+100

West

7

0

South

+0

East

R

2

1 (Std)

North

+100

West

8

8

0

South

+0

East

S

3

1 (Std)

North

+100

West

9

9

0

South

+0

East

T

4

1 (Std)

North

+100

West

A

0

1 (Custom)

U

5

1 (Std)

North

+100

West

B

1

1 (Custom)

V

6

1 (Std)

North

+100

West

C

2

1 (Custom)

W

7

1 (Std)

North

+100

West

D

3

1 (Custom)

X

8

1 (Std)

North

+100

West

E

4

1 (Custom)

Y

9

1 (Std)

North

+100

West

F

5

1 (Custom)

Z

space

1 (Std)

North

+100

West

G

6

1 (Custom)

Note: the ASCII characters A–K are not used in address bytes 4–6.

For example, for a station at a latitude of 33 degrees 25.64 minutes north, in
the western hemisphere, with longitude offset +0 degrees, and transmitting
standard message identifier bits 1/0/0, the encoding of the first 6 bytes of the
Destination Address field is as follows:
Destination Address Byte:

1

2

3

4

5

6

Latitude Digit

3

3

2

5

6

4

Message Bits

Message
Bit A
= 1 (Std)

Message
Bit B
=0

Message
Bit C
=0
North

N/S Indicator

+0

Long Offset

West

W/E Indicator
Dest Address
(ASCII Char)

% Document Version 1.0.1: 29 August 2000

S

3

2

U

6

T


Mic-E Messages

The first three bytes of the Destination Address field contain three message
identifier bits: A, B and C. These bits allow one of 15 message types to be
specified:

\begin{itemize}

\item 7 Standard messages
\item 7 Custom messages
\item 1 Emergency message

\end{itemize}

For the 7 Standard messages, one or more of the message identifier bits is a
1, shown in the Mic-E Destination Address Field Encoding table as 1 (Std).
For the 7 Custom messages, one or more of the message identifier bits is a 1,
shown in the Mic-E Destination Address Field Encoding table as 1 (Custom).
For the Emergency message, all three message identifier bits are 0.
The following table shows the encoding of Mic-E message types, for all
combinations of the A/B/C message identifier bits:
Mic-E Message Types
Standard Mic-E
Message Type

Custom Mic-E
Message Type

A

B

C

1

1

1

M0: Off Duty

C0: Custom-0

1

1

0

M1: En Route

C1: Custom-1

1

0

1

M2: In Service

C2: Custom-2

1

0

0

M3: Returning

C3: Custom-3

0

1

1

M4: Committed

C4: Custom-4

0

1

0

M5: Special

C5: Custom-5

0

0

1

M6: Priority

C6: Custom-6

0

0

0

Emergency

The Standard messages and the Emergency message have the same meaning
for all APRS stations. The Custom messages may be assigned any arbitrary
meaning.
Note: Support for Custom messages is optional. Original Mic-E units do not
support Custom messages.
Note: If the A/B/C message identifier bits contain a mixture of Standard 1s
and Custom 1s, the message type is “unknown”.


Some examples of message type encoding:
First 3
Destination
Address Bytes

Message Identifier
Bits A/B/C

Message
Type

Message

S32

Standard 1 / 0 / 0

Standard

M3: Returning

F2D

Custom 1 / 0 / Custom 1

Custom

C2: Custom-2

234

0/0/0

Emergency

Emergency

The SSID in the Destination Address field of a Mic-E packet is coded to
specify either a conventional digipeater VIA path (contained in the
Digipeater Addresses field of the AX.25 frame), or one of 15 generic APRS
digipeater paths. See Chapter 4: APRS Data in the AX.25 Destination and
Source Address Fields.

Destination
Address
SSID Field

The Information field is used to complete the Position Report that was begun
in the Destination Address field. The encoding used is different from the
destination address since the content is not constrained to be printable,
shifted 7-bit ASCII, as it is in the address. However, full 8-bit binary is not
used — all values are offset by 28 and further operations (described below)
are performed on some of the values to make almost all of the data printable
ASCII.

Mic-E Information
Field

The format of the Information field is as follows:
Mic-E Data — INFORMATION FIELD Format

Bytes:

Longitude

Speed and Course

Data
Type
ID

d+28

m+28

h+28

SP+28

DC+28

SE+28

1

1

1

1

1

1

1

Information Field
Data

Symbol
Code

Sym
Table
ID

Mic-E Telemetry Data

1

1

n

Mic-E Status Text

The first 9 bytes of the Information field contain the APRS Data Type
Identifier, longitude, speed, course and symbol data.
The APRS Data Type Identifier is one of:
‘
Current GPS data
(but not used in Kenwood TM-D700 radios) .
'
Old GPS data
(or Current GPS data in Kenwood TM-D700 radios).
0x1c Current GPS data (Rev. 0 beta units only).
0x1d Old GPS data (Rev. 0 beta units only).



IMPORTANT NOTE: As explained in detail below, some of the bytes in
the Information field are non-printing ASCII characters. In certain
circumstances (such as incorrect TNC setup or inappropriate software), some
of these non-printing characters may be dropped, making the Information
field appear shorter than it really is. This will lead to incorrect decoding. (In
particular, if the Information field appears to be less than 9 bytes long, the
packet must be ignored).

Longitude Degrees
Encoding

The d+28 byte in the Information field contains the encoded value of the
longitude degrees, in the range 0–179 degrees.
(Note that for longitude values in the range 0–9 degrees, the longitude offset
is +100 degrees):
Mic-E Longitude Degrees Encoding
Long
Deg
0
1
2
3
4
5
6
7
8
9
10
11
12
…
97
98
99

ASCII
Char
v
w
x
y
z
{
|
}
~

d+28

Long
Offset

Long
Deg

DEL

118
119
120
121
122
123
124
125
126
127

+100
+100
+100
+100
+100
+100
+100
+100
+100
+100

100
101
102
103
104
105
106
107
108
109

&
'
(

38
39
40

+0
+0
+0

}
~

125
126
127

+0
+0
+0

110
111
112
…
177
178
179

DEL

ASCII
Char
l
m
n
o
p
q
r
s
t
u

d+28

Long
Offset

108
109
110
111
112
113
114
115
116
117

+100
+100
+100
+100
+100
+100
+100
+100
+100
+100

&
'
(

38
39
40

+100
+100
+100

i
j
k

105
106
107

+100
+100
+100


Note from the table that the encoding is split into four separate pieces:

\begin{itemize}
\item 0–9 degrees: d+28 is in the range 118–127 decimal, corresponding to
the ASCII characters v to DEL.
Important Note: The longitude offset is set to +100 degrees when
the longitude is in the range 0–9 degrees.

\item 10–99 degrees: d+28 is in the range 38–127 decimal (corresponding
to the ASCII characters & to DEL), and the longitude offset is +0
degrees.

\item 100–109 degrees: d+28 is in the range 108–117 decimal,
corresponding to the ASCII characters l (lower-case letter “L”) to
DEL, and the longitude offset is +100 degrees.

\item 110–179 degrees: d+28 is in the range 38–127 decimal
(corresponding to the ASCII characters & to DEL), and the longitude
offset is +100 degrees.


\end{itemize}

Thus the overall range of valid d+28 values is 38–127 decimal
(corresponding to ASCII characters & to DEL).
All of these characters (except DEL, for 9 and 99 degrees) are printable
ASCII characters.
To decode the longitude degrees value:
1. subtract 28 from the d+28 value to obtain d.
2. if the longitude offset is +100 degrees, add 100 to d.
3. subtract 80 if 180 ˜ d ˜ 189
(i.e. the longitude is in the range 100–109 degrees).
4. or, subtract 190 if 190 ˜ d ˜ 199.
(i.e. the longitude is in the range 0–9 degrees).
Longitude Minutes
Encoding

The m+28 byte in the Information field contains the encoded value of the
longitude minutes, in the range 0–59 minutes:
Mic-E Longitude Minutes Encoding
Long
Mins
0
1
2
3
4
5
6
7
8
9

ASCII
Char
X
Y
Z
[
\
]
^
_
‘
a

m+28

Long
Mins

88
89
90
91
92
93
94
95
96
97

10
11
12
13
14
…
56
57
58
59

ASCII
Char
&
'
(
)
*

m+28

T
U
V
W

84
85
86
87

38
39
40
41
42

Note from the table that the encoding is split into two separate pieces:
•

0–9 minutes: m+28 is in the range 88–97 decimal, corresponding to
the ASCII characters X to a.

•

10–59 minutes: m+28 is in the range 38–87 decimal (corresponding
to the ASCII characters & to W).

Thus the overall range of valid m+28 values is 38–97 decimal (corresponding
to ASCII characters & to a). All of these characters are printable ASCII
characters.
To decode the longitude minutes value:
1. subtract 28 from the m+28 value to obtain m.
2. subtract 60 if m ™ 60.
(i.e. the longitude minutes is in the range 0–9).

Longitude
Hundredths of
Minutes Encoding

The h+28 byte in the Information field contains the encoded value of the
longitude hundredths of minutes, in the range 0–99 minutes. This byte takes
a value in the range 28 decimal (corresponding to 0 hundredths of a minute)
through 127 decimal (corresponding to 99 hundredths of a minute).
To decode the longitude hundredths of minutes value, subtract 28 from the
h+28 value.
All of the possible values are printable ASCII characters (except 0–3 and 99
hundredths of a minute).

Speed and Course
Encoding

The speed and course of a station are encoded in 3 bytes, designated SP+28,
DC+28 and SE+28.
The speed is in the range 0–799 knots, and the course is in the range 0–360
degrees (0 degrees represents an unknown or indefinite course, and 360
degrees represents due north).
The encoded speed and course are spread over the three bytes, as follows:
Speed

Course

Encoded Speed
(hundreds/tens of knots)

Encoded Speed (units) and
Encoded Course
(hundreds of degrees)

Encoded Course
(tens/units)

SP+28

DC+28

SE+28


SP+28 Encoding

The SP+28 byte contains the encoded speed, in hundreds/tens of knots,
according to this table:
SP+28 Speed Encoding (hundreds/tens of knots)
Speed
knots
0-9
10-19
20-29
30-39
40-49
50-59
60-69
70-79
80-89
90-99
100-109
110-119
120-129
130-139
140-149
150-159
160-169
170-179
180-189
190-199

l
m
n
o
p
q
r
s
t
u
v
w
x
y
z
{
|
}
~

ASCII
Char
0x1c
0x1d
0x1e
0x1f

DEL

V
!
"
#
$
%
&
'
(
)
*
+
,
.
/

SP +28
108
109
110
111
112
113
114
115
116
117
118
119
120
121
122
123
124
125
126
127

28
29
30
31
32
33
34
35
36
37
38
39
40
41
42
43
44
45
46
47

Speed
knots
200-209
210-219
220-229
230-239
240-249
250-259
260-269
270-279
280-289
290-299
300-310
310-320
…
730-739
740-749
750-759
760-769
770-779
780-789
790-799

ASCII
Char
0
1
2
3
4
5
6
7
8
9
:
;

SP +28

e
f
g
h
i
j
k

101
102
103
104
105
106
107

48
49
50
51
52
53
54
55
56
57
58
59

Note: The ASCII characters shown in white on a black background are nonprinting characters.
Note: For speeds in the range 0–199 knots, there are two encoding schemes
in existence. Hence there are two columns for the ASCII character, and two
columns for the corresponding SP+28 byte values.
For example, for a speed of 73 knots (i.e. in the range 70–79), the SP+28 byte
may contain either s or #, depending on the encoding method used. Both are
equally valid.
The decoding algorithm described later handles either of these encoding
schemes.


DC+28 Encoding

The DC+28 byte contains the encoded units of speed, plus the encoded course
in hundreds of degrees:

DC+28 Speed / Course Encoding (units of knots/hundreds of degrees)
Knots
(units)

Course
(deg)

0
0
0
0

0-99
100-199
200-299
300-360

V
!

1
1
1
1

0-99
100-199
200-299
300-360

*
+
,
-

2
2
2
2

0-99
100-199
200-299
300-360

3
3
3
3
4
4
4
4

Knots
(units)

Course
(deg)

32
33
34
35

28
29
30
31

5
5
5
5

0-99
100-199
200-299
300-360

ASCII
Char
N
R
S
O
P
T
U
Q

&
'
(
)

42
43
44
45

38
39
40
41

6
6
6
6

0-99
100-199
200-299
300-360

\
]
^
_

X
Y
Z
[

92
93
94
95

88
89
90
91

4
5
6
7

0
1
2
3

52
53
54
55

48
49
50
51

7
7
7
7

0-99
100-199
200-299
300-360

f
g
h
i

b
c
d
e

102
103
104
105

98
99
100
101

0-99
100-199
200-299
300-360

>
?
@
A

:
;
<
=

62
63
64
65

58
59
60
61

8
8
8
8

0-99
100-199
200-299
300-360

p
q
r
s

l
m
n
o

112
113
114
115

108
109
110
111

0-99
100-199
200-299
300-360

H
I
J
K

D
E
F
G

72
73
74
75

68
69
70
71

9
9
9
9

0-99
100-199
200-299
300-360

z
{
|
}

v
w
x
y

122
123
124
125

118
119
120
121

"
#

ASCII
Char
0x1c
0x1d
0x1e
0x1f

DC +28

DC +28
82
83
84
85

78
79
80
81

Note: The ASCII characters shown in white on a black background are nonprinting characters.

Note: There are two encoding schemes in existence for the DC+28 byte.

Hence there are two columns for the ASCII character, and two columns for
the corresponding DC+28 byte values.

For example, for a speed of 73 knots (i.e. units=3) and a bearing of 294
degrees (i.e. in the range 200–299), the DC+28 byte may contain either @ or
<, depending on the encoding method used. Both are equally valid.
The decoding algorithm described later handles either of these encoding
schemes.



SE+28 Encoding

The SE+28 byte contains the encoded tens and units of degrees of the course:
SE+28 Course Encoding (tens/units of degrees)
Course
(deg)
0
1
2
3
4
5
6
7
8
9
10
11
12
13
14

Example of Mic-E
Speed and Course
Encoding

Decoding the
Speed and Course

ASCII
Char
0x1c
0x1d
0x1e
0x1f
V
!
"
#
$
%
&
'
(
)
*

m+28

Long
Mins

28
29
30
31
32
33
34
35
36
37
38
39
40
41
42

15
16
17
18
19
…
91
92
93
94
95
96
97
98
99

ASCII
Char
+
,
.
/

m+28

w
x
y
z
{
|
}
~

119
120
121
122
123
124
125
126
127

DEL

43
44
45
46
47

For a speed of 86 knots and a course of 194 degrees, the encoding is:
SP+28:

The speed is in the range 80–89 knots. From the SP+28 encoding
table, the SP+28 byte may be either t or $.

DC+28:

The units of speed are 6, and the course is in the range 100–199
degrees. From the DC+28 encoding table, the DC+28 byte may be
either ] or Y.

SE+28:

The course in tens and units of degrees is 94. From the SE+28
encoding table, the SE+28 byte will be z.

To decode the speed and course:
SP+28:

To obtain the speed in tens of knots, subtract 28 from the SP+28
value and multiply by 10.

DC+28:

Subtract 28 from the DC+28 value and divide the result by 10. The
quotient is the units of speed. The remainder is the course in
hundreds of degrees.

SE+28:

To obtain the tens and units of degrees, subtract 28 from the SE+28
value.

Finally, make these speed and course adjustments:
•

If the computed speed is ™ 800 knots, subtract 800.

•

If the computed course is ™ 400 degrees, subtract 400.
