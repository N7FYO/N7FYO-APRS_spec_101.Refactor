\chapter{Chapter 2: APRS Design Philosophy}


\section{Net Cycle Time}

It is important to note that APRS is primarily a real-time, tactical
communications tool, to help the flow of information for things like special
events, emergencies, Skywarn, the Emergency Operations Center and just
plain in-the-field use under stress. But like the real world, for 99\% of the
time it is operating routinely, waiting for the unlikely serious event to
happen.

Anything which is done to enhance APRS must not undermine its ability to
operate in local areas under stress. Here are the details of that philosophy:

\begin {enumerate}

\item APRS uses the concept of a “net cycle time”. This is the time within
which a user should be able to hear (at least once) all APRS stations
within range, to obtain a more or less complete picture of APRS activity.
The net cycle time will vary according to local conditions and with the
number of digipeaters through which APRS data travels.

\item The objective is to have a net cycle time of 10 minutes for local use. This
means that within 10 minutes of arrival on the scene, it is possible to
captured the entire tactical picture.

\item  All stations, even fixed stations, should beacon their position at the net
cycle time rate. In a stress situation, stations are coming and going all the
time. The position reports show not only where stations are without
asking, but also that they are still active.

\item It is not reasonable to assume that all APRS users responding to a stress
event understand the ramifications of APRS and the statistics of the
channel — user settings cannot be relied on to avoid killing a stressed
net. Thus, to try to anticipate when the channel is under stress, APRS
automatically adjusts its net cycle time according to the number of
digipeaters in the UNPROTO path:

\begin{itemize}

\item Direct operation (no digipeaters): 10 minutes (probably an
event).

\item Via one digipeater hop: 10 minutes (probably an event).

\item Via two digipeater hops: 20 minutes.

\item Via three or more digipeater hops: 30 minutes.

\end{itemize}

\item Since almost all home stations set their paths to three or more
digipeaters, the default net cycle time for routine daily operation is
30 minutes. This should be a universal standard that everyone can bank
on -- if you routinely turn on your radio and APRS and do nothing
else, then in 30 minutes you should have virtually the total picture
of all APRS stations within range.

\item Since knowing where the digipeaters are located is fundamental to APRS
connectivity, digipeaters should use multiple beacon commands to
transmit position reports at different rates over different paths; i.e. every
10 minutes for sending position reports locally, and every 30 minutes for
sending them via three digipeaters (plus others rates and distances as
needed).

\item If the net cycle time is too long, users will be tempted to send queries for
APRS stations. This will increase the traffic on the channel
unnecessarily. Thus the recommended extremes for net cycle time are 10
and 30 minutes — this gives network designers the fundamental
assumptions for channel loading necessary for good engineering design.

\end{enumerate}

\section {Packet Timing}

Since APRS packets are error-free, but are not guaranteed delivery, APRS
transmits information redundantly. To assure rapid delivery of new or
changing data, and to preserve channel capacity by reducing interference
from old data, APRS should transmit new information more frequently than
old information.

There are several algorithms in use to achieve this:

\begin{itemize}


\item \textbf{Decay Algorithm --} Transmit a new packet once and n seconds later.
Double the value of n for each new transmission. When n reaches the
net cycle time, continue at that rate. Other factors besides
“doubling” may be appropriate, such as for new message lines.

\item \textbf{Fixed Rate --} Transmit a new packet once and n seconds later. Transmit
it x times and stop.

\item \textbf{Message-on-Heard --} Transmit a new packet according to either
algorithm above. If the packet is still valid, and has not been
acknowledged, and the net cycle time has been reached, then the
recipient is probably not available. However, if a packet is then
subsequently heard from the recipient, try once again to transmit the
packet.

\item \textbf{Time-Out --} This term is used to describe a time period beyond which it
is reasonable to assume that a station no longer exists or is off the air if
no packets have been heard from it. A period of 2 hours is suggested as
the nominal default timeout. This time-out is not used in any transmitting
algorithms, but is useful in some programs to decide when to cease
displaying stations as “active”. Note that on HF, signals come and go, so
decisions about activity may need to be more flexible.


\end{itemize}

\section{Generic Digipeating}


The power of APRS in the field derives from the use of generic digipeating,
in that packets are propagated without a priori knowledge of the network.
There are six powerful techniques which have evolved since APRS was
introduced in 1992:

\begin{enumerate}

\item \textbf{RELAY --} Every VHF APRS TNC is assumed to have an alias of
RELAY, so that anyone can use it as a digipeater at any time.

\item \textbf{ECHO --} HF stations use the alias of ECHO as an alternative to
RELAY. (However, bearing in mind the nature of HF propagation, this
has the potential of causing interference over a wide area, and should
only be used sparingly by mobile stations).

\item \textbf{WIDE --} Every high-site digipeater is assumed to have an alias of WIDE
for longer distance communications.

\item \textbf{TRACE --} Every high-site digipeater that is using callsign substitution
is assumed to have the alias of TRACE. These digipeaters self-identify
packets they digipeat by inserting their own call in place of RELAY,
WIDE or TRACE.

\item \textbf{WIDEn-N --} A digipeater that supports WIDEn-N digipeating will
digipeat any WIDEn-N packet that is “new” and will subtract 1 from the
SSID until the SSID reaches –0. The digipeater keeps a copy or a
checksum of the packet and will not digipeat that packet again within
(typically) 28 seconds. This considerably reduces the number of
superfluous digipeats in areas with many digipeaters in radio range of
each other.

\item \textbf{GATE --} This generic callsign is used by HF-to-VHF Gateway
digipeaters. Any packet heard on HF via GATE will be digipeated locally
on VHF. This permits local networks to keep an eye on the national and
international picture.

\end{enumerate}

\section{Communicating Map Views Unambiguously}

APRS is a tactical geographical system. To maximize its operational
effectiveness and minimize confusion between operators of different
systems, users need to have an unambiguous way to communicate to others
the “location” and “size” (or area of coverage) of any map view.
The APRS convention is by reference to a center and range which specify
the geographical center and approximate radius of a circle that will fit in the
map view independent of aspect ratio. The radius of the circle (in nautical
miles, statute miles or km) is known as the “range scale”. This convention
gives all users a simple common basis for describing any specific map view
to others over any communications medium or program.
