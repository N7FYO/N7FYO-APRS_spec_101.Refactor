\chapter{Appendix 5: Glossary}


\paragraph{Altitude}

1. In Mic-E format, the altitude in meters relative to 10km below mean sea level.
2. In Comment text, the altitude in feet above mean sea level.

Announcement
Announcement Identifier
Antenna Height

\paragraph {APRS}

An APRS message that is repeated a few times an hour, perhaps for several days.
A single letter A-Z that identifies a particular announcement.
In NMEA sentences, the height of the antenna in meters relative to mean sea level.
(The antenna height in GPS NMEA sentences fluctuates wildly because of Selective
Availability, and should only be used if DGPS correction is applied).
Automatic Position Reporting System.

APRS Data

The data that follows the APRS Data Type Identifier in the AX.25 Information field and
precedes the APRS Data Extension.

APRS Data Extension

A 7-byte extension to APRS Data. The Data Extension includes one of Course/Speed,
Wind Direction/Wind Speed, Station Power/Antenna Effective Height/Gain/Directivity,
Pre-Calculated Radio Range, DF Signal Strength/Effective Antenna Height/Gain, Area
Object Descriptor.

APRS Digipeater Path

A digipeater path via repeaters with RELAY, WIDE and related aliases. Used in Mic-E
compressed location format.

APRS Data Type Identifier

The single-byte identifier that specifies what kind of APRS information is contained in
the AX.25 Information field.

Area Object

A user-defined graphic object (circle, ellipse, triangle, box and line).

ASCII

American Standard Code for Information Interchange. A 7-bit character code
conforming to ANSI X3.4 (1968) — see Appendix 3 for character definitions.

AX.25

Amateur Packet-Radio Link-Layer Protocol.

Base 91

Number base used to ensure that numeric values are transmitted as printable ASCII
characters. To obtain the character string corresponding to a numeric value, divide the
value progressively by decreasing powers of 91, and add 33 decimal to the result at
each step. Printable characters are in the range !..{. Used in compressed lat/long and
altitude computation.

Bulletin

An APRS message that is repeated several times an hour, for a small number of
hours. A General Bulletin is addressed to no-one in particular. A Group Bulletin is
addressed to a named group (e.g. WX).

Bulletin Identifier
Destination Address field
DF Report
DGPS
DHM
DHMz
Digipeater
Digipeater Addresses field
Directivity
DX Cluster
ECHO
Effective Antenna Height

Document Version 1.0.1: 29 August 2000

A single digit 0-9 that identifies a particular bulletin.
The AX.25 Destination Address field, which can contain an APRS destination callsign
or Mic-E encoded data.
A report containing DF bearing and range.
Differential GPS. Used to overcome the errors arising from Selective Availability.
7-character timestamp: day-of-the-month, hour, minute, zulu or local time.
7-character timestamp: day-of-the-month, hour, minute, zulu only.
A station that relays AX.25 packets. A chain of up to 8 digipeaters may be specified.
The AX.25 field containing 0–8 digipeater callsigns (or aliases).
The favored direction of an antenna. Used in the PHG Data Extension.
A network host that collects and disseminates user reports of DX activity.
A generic APRS digipeater callsign alias, for an HF digipeater.
The height of an antenna above the local terrain (not above sea level). A first-order
indicator of the antenna’s effectiveness in the local area. Used in the PHG Data

APRS Protocol Reference — APRS Protocol Version 1.0



Extension.
ERP

Effective Radiated Power. Used in Status Reports containing Beam Heading and
Power data (typically for meteor scatter use).

FCS

Frame Check Sequence. A sequence of 16 bits that follows the AX.25 Information
field, used to verify the integrity of the packet.

GATE

A gateway between HF and VHF APRS networks. Used primarily to relay longdistance HF APRS traffic onto local VHF networks.

GGA Sentence

A standard NMEA sentence, containing the receiving station’s lat/long position and
antenna height relative to mean sea level, and other data.

GLL Sentence

A standard NMEA sentence, containing the receiving station’s lat/long position and
other data.

GMT

Greenwich Mean Time (=UTC=zulu).

GPS

Global Positioning System. A global network of 24 satellites that provide lat/long and
antenna height of a receiving station.

GPSxyz

An APRS destination callsign that specifies a display symbol from either the Primary
Symbol Table or the Alternate Symbol Table. Some symbols from the Alternate
Symbol Table can be overlaid with a digit or a letter. Used by trackers that cannot
specify the symbol in the AX.25 Information field.

GPSCnn

An APRS destination callsign that specifies a display symbol from the Primary Symbol
Table. The symbol can not be overlaid. Used by trackers that cannot specify the
symbol in the AX.25 Information field.

GPSEnn

An APRS destination callsign that specifies a display symbol from the Alternate
Symbol Table. The symbol can not be overlaid. Used by trackers that cannot specify
the symbol in the AX.25 Information field.

HMS

1. In NMEA sentences, a 6-character timestamp: hour, minute, second UTC.
2. In APRS Data, a 7-character timestamp: hour, minute, second, zulu or local.

ICQ
IGate
Information field
Item
Item Report
Killed Object
knots
KPC-3
Longitude Offset
LORAN
Maidenhead Locator
MDHM
Message
Message Acknowledgement
Message Group
Message Identifier
Mic-E

International CQ chat.
A gateway between a VHF and/or HF APRS network and the Internet.
The AX.25 Information field containing APRS information.
A type of display object.
A report containing the location of an APRS Item.
An Object that an APRS user has assumed control of.
International nautical miles per hour.
A Terminal Node Controller from Kantronics Co Inc.
An offset of +100 degrees longitude (used in Mic-E longitude computation).
Long Range Navigation System (a terrestrial precursor to GPS).
A 4- or 6-character grid locator specifying a station’s position.
8-byte timestamp: month, day, hour, minute (used in positionless weather station
reports).
A one-line text message addressed to a particular station.
An optional acknowledgement of receipt of a message.
A user-defined group to receive messages.
A 1–5 character message identifier (typically a line number).
Originally Microphone Encoder, a unit that encodes location, course and speed
information into a very short packet, for transmission when releasing the microphone
PTT button. The Mic-E encoding algorithm is now used in other devices (e.g. in the


PIC-E and the Kenwood TH-D7/TM-D700 radios).
Mic-E Message Identifier
Mic-E Message Code

A 3-bit identifier (A/B/C) specifying a standard Mic-E message or custom message
code.
A 3-bit code specifying a Standard or Custom Mic-E message.

MIM

Micro Interface Module. A complete telemetry TNC transmitter on a chip.

mph

miles per hour.

Net Cycle Time

NMEA

NMEA (Received) Sentence

NRQ
Null Position
NWS
Object
Object Report
PHG
PIC
PIC-E
Position Ambiguity

The time within which it should be possible to gain the complete picture of APRS
activity (typically 10, 20 or 30 minutes, depending on the number of digipeaters
traversed and local conditions). Stations should not transmit status or position
information more frequently unless mobile, or in response to a Query.
National Marine Electronic Association (United States). Producer of the NMEA 0183
Version 2.0 specification that governs the format of Received Sentences from
navigation equipment (such as GPS and LORAN receivers). See Appendix 6 for a
reference to NMEA sentence formats.
The ASCII data stream received from navigation equipment (such as GPS receivers)
conforming to the NMEA 0182 Version 2.0 specification. APRS supports five NMEA
Sentences: GGA, GLL, RMC, VTG and WPT.
Number/Rate/Quality. A measure of confidence in DF Bearing reports.
Default position to be reported if the actual position is unknown or indeterminate. The
null position is 0° 0' 0" north, 0° 0' 0" west.
National Weather Service (United States).
A display object that is (usually) not a station. For example, a weather front or a
marathon runner.
A report containing the position of an object, with optional timestamp and APRS Data
Extension.
APRS Data Extension specifying Power, Effective Antenna Height/Gain/Directivity.
Programmable Interface Controller.
A PIC implementation of the Mic-E microphone encoder.
A reduction in the accuracy of APRS position information (implemented by replacing
low-order lat/long digits with spaces). Used when the exact position is not known.

Position Report

A report containing lat/long position, optionally with timestamp and Data Extension.

Pre-Calculated Radio Range

A station’s estimate of omni-directional radio range (in miles). Used in compressed
lat/long format.

Query
Range Circle
RELAY
Response
RMC Sentence
RTCM
Selective Availability

A request for information. Queries may be addressed to stations in general or to
specific stations.
Usable radio range (in miles), computed from PHG data.
A generic APRS digipeater callsign alias, for a VHF/UHF digipeater with limited local
coverage.
A reply to a query.
A standard NMEA sentence, containing the receiving station’s lat/long position, course
and speed, and other data.
Radio Technical Commission for Maritime Services. The RTCM SC104 data format
specification describes the requirements for differential GPS data correction.
Deliberate GPS position dithering, introducing significant received position errors in
latitude, longitude and antenna height. Errors can be greatly reduced with differential
GPS.

Sentence

See NMEA (Received) Sentence.

Signpost

A special signpost icon that displays user-defined variable information (such as a

speed limit or mileage) as an overlay.
Skywarn
Source Address Field
Source Path Header
SPCL
SSID

Station Capabilities
Status Report
Switch Stream Character
Symbol

Symbol Code
Symbol Table Identifier

Target Footprint
TH-D7
TM-D700
Third Party Network
Third-Party Header
TNC
Trace

A weather spotter initiative coordinated by the United States National Weather
Service.
The AX.25 Source Address field, containing the callsign of the originating station. A
non-zero SSID specifies a display symbol.
The digipeater path followed prior to a packet entering a Third-Party Network.
A generic APRS destination callsign used for special stations.
Secondary Station Identifier. A number in the range 0-15, as an adjunct to an AX.25
address. If the SSID in a source address is non-zero, it specifies a display symbol.
(This is used when the station is unable to specify the symbol in the AX.25 Destination
Address field or Information field).
A list of station characteristics that is sent in reply to a query.
A report containing station status information (and optionally a Maidenhead locator).
A character normally used for switching TNC channels.
A display icon. Consists of a Symbol Table Identifier/Symbol Code pair. Generically,
/$ represents a symbol from the Primary Symbol Table, and \$ represents a symbol
from the Alternate Symbol Table.
A code for a symbol within a Symbol Table.
An ASCII code specifying the Primary Symbol Table (/) or Alternate Symbol Table (\).
The Symbol Table Identifier is also implicit in GPSCnn and GPSEnn destination
callsigns.
A target area for queries. The querying station asks for responses from stations within
a specified number of miles of a lat/long position.
A combined VHF/UHF handheld radio and APRS-compatible TNC from Kenwood.
A combined VHF/UHF mobile radio and APRS-compatible TNC from Kenwood.
A non-APRS network that does not understand AX.25 addresses (e.g. the Internet).
A Path Header with the Third-Party Network Identifier and the callsign of the receiving
gateway inserted.
Terminal Node Controller. A combined AX.25 packet assembler/disassembler and
modem.
An APRS query that asks for the route taken by a packet to a specified station.

TRACE

A generic digipeater callsign alias, for digipeaters that performs callsign substitution.
These digipeaters self-identify packets they digipeat, by inserting their own callsign in
place of RELAY,WIDE or TRACE.

Tracker

A unit comprising a GPS receiver (to obtain the current geographical position) and a
radio transmitter (to transmit the position to other stations).

Tunneling

UI-Frame
UNPROTO Path
UTC
VTG Received Sentence
WIDE
WIDEn-N

Passing APRS AX.25 traffic through a third-party network that does not understand
AX.25 addressing. The AX.25 addresses are carried as data (in the Source Path
Header) through the tunneled network.
AX.25 Unnumbered Information frame. APRS uses only UI-frames — that is, it
operates entirely in connectionless (UNPROTO) mode.
The digipeater path to the destination callsign.
Coordinated Universal Time (=zulu=GMT).
A standard NMEA sentence, containing the receiving station’s course and speed.
A generic APRS digipeater callsign alias, for a digipeater with wide area coverage.
A generic APRS digipeater callsign alias, for a digipeater with wide area coverage
(N=0-7). As a packet passes through a digipeater, the value of N is decremented by 1
until it reaches zero. The digipeater keeps a record of each packet (or its FCS) as it
passes through, and will not digipeat the packet again if it has digipeated it already
within the last 28 seconds.
WPT Sentence
WX
Ziplan
Zulu

A standard NMEA sentence, containing waypoints.
Weather.
A cheap twisted-pair LAN connecting PCs via their serial I/O ports. Designed for use in
emergency situations.
UTC/GMT.

Units Conversion Table
To convert from
feet
meters
miles
km
miles
nautical miles
miles per hour (mph)

to
meters

km

knots

knots

meters / second

meters / second

1.609344
1.609344
0.8689762

miles

miles per hour (mph)

miles per hour (mph)

0.3048

miles
nautical miles

0.8689762
0.8689762
0.8689762
0.51444’

knots
meters / second
miles per hour (mph)

divide by

0.3048

feet

knots

meters / second

multiply by

0.51444’
0.44704
0.44704

Fahrenheit / Celsius Temperature Conversion Equations
F = ( C x 1.8 ) + 32
C = ( F – 32 ) x 5
9
