\chapter{Appendix 7: Document Release History}


Date

Doc
Version

Status / Major Changes

10 Oct 1999

1.0 (Draft)

Protocol Version 1.0. First public draft release.

3 Dec 1999

1.0.1g

Protocol Version 1.0. Second public draft release. Much extended, incorporating packet format
layouts, APRS symbol tables, compressed data format, Mic-E format, telemetry format.

30 Apr 2000

1.0.1m

Protocol Version 1.0. Third public draft release.
Major additions/changes to the draft 1.0.1g specification:

\begin{itemize}
Added a section on Map Views and Range Scale.
Changed Destination Address SSID description (specifying generic APRS digipeater paths)
to apply to all packets, not just Mic-E packets.
Changed APRS destination “callsigns” to “destination addresses”.
Added TEL* to the list of generic destination addresses.
Added brief explanations of how several generic destination addresses are used.
Added “Grid-in-To-Address” (but marked as obsolete).
Extended the description of the Comment field, with pointers to what can appear in the field.
Added explanation of base 91.
Added paragraph on lack of consistency in on-air units, and default GPS datum = WGS84.
APRS Data Type Identifiers Table:
marked Shelter Data and Space Weather as reserved DTIs.
marked the - DTI as unused (previously erroneously allocated to Killed Objects).
marked the ' DTI to mean Current Mic-E data in Kenwood TM-D700 radios.
marked the ‘ DTI as not used in Kenwood TM-D700 radios.
Position Ambiguity: need only be specified in the latitude — the longitude will have the same
level of ambiguity.
Added the options of .../... and V V V /VV V V to express unknown course/speed.
Added DFS parameter table.
Added Quality table for BRG/NRQ data.
Position, DF and Compressed Report formats: split the format diagrams into two parts (with
and without timestamps).
DF Reports: added notes:
BRG/NRQ data is only valid when the symbol is /\.
CSE=000 means the DF station is fixed, CSE non-zero means the station is moving.
Compressed position reports: corrected the multiplication/division constants for encoding/
decoding.
Mic-E chapter rewritten and expanded. Emphasized the need to ensure that non-printing
ASCII characters are not dropped. Corrected the Mic-E telemetry data format.
Expanded the introductory description of Objects/Items. All Objects must have a timestamp.
Added Area Object Extended Data field to Object and Item format diagrams.
Added Object/Item format diagrams with compressed location data.
Killed Objects/Items: now indicated by underscore after the name.
Re-categorized weather reports: Raw, Positionless and Complete.
Added a statement that temperatures below zero are expressed as -01 to -99.
Added the options of ... and V V V to express unknown weather parameter values.
Corrected the storm data format. Also, central pressure is now /ppppp (tenths of millibar).
Corrected the telemetry parameter data (now APRS messages instead of AX.25 UI beacons).
Added optional comment field to the Telemetry (T) format.
Added a section describing the handling of multiple message acknowledgements.
Added a section on NTS radiograms.
Added Bulletin/Announcement implementation recommendations.
Queries and Responses:
Query Names (e.g. APRSD): all upper-case.
A queried station need not respond if it has no relevant information to send.
A queried station should ignore any query type that it does not recognize.
APRSH: callsigns must be padded to 9 characters.
Added PING as a synonym of APRST.
Extended meteor scatter ERP beyond 810 watts, and added a lookup table.
Maidenhead Locator: all letters must be transmitted in upper case, but may be received in
either upper or lower case.
Changed the definition of non-APRS packets — these are not APRS Status Messages, but
may optionally be treated as such.
APRS Symbols chapter substantially rewritten..
Added section on Symbol Precedence (where more than one symbol appears in an APRS
packet).
Clarified some of the descriptions in the APRS Symbol Tables.
Added overlay capability to the \a symbol (ARES/RACES etc).
Separated the 7-bit ASCII table from the Dec/Hex (0x80-0xff) conversion table.
Added several new entries and a units conversion table to the Glossary.
Added new references to NMEA sentence formats and Maidenhead Locator formats.


Date
29 Aug 2000

Doc
Version
1.0.1

Status / Major Changes
Protocol Version 1.0. Approved public release.

Minor additions/changes to the draft 1.0.1m specification:
• Added Foreword.
• Replaced section on Map Views and Range Scale.
• APRS Software Version No: added APDxxx (Linux aprsd server).
• APRS Data Type Identifier: Designated [ as Maidenhead grid locator (but noted as obsolete).
• Position Ambiguity: added a bounding box example.
• Compressed Position Formats: for course/speed, corrected the range of possible values of the
“c” byte to 0–89.
• Mic-E: replaced the latitude example table, to show more explicitly how the N/S/E/W/Long
offset bits are encoded.
• Mic-E: removed the paragraph stating that there must be a space between the altitude and
comment text — no space is required.
• Mic-E: removed the note on inaccurate altitude data, as GPS Selective Availability has been
switched off.
• Object Reports: added timestamps to some of the examples (an Object Report must always
have a timestamp).
• Signposts: can be Objects or Items.
• Storm Data: changed central pressure format to /pppp (i.e. to the nearest millibar/hPascal).
• Storm Data: Hurricane Brenda examples: inserted a leading zero in the central pressure field
(central pressure is 4 digits).
• Telemetry Data: Added MIC as an alternative form of Sequence Number. MIC may or may
not be followed by a comma.
• Messages: added the reject message format.
• Appendix 1: Agrelo format: changed the separator between Bearing and Quality to /.
• Symbol Table: changed /( symbol from “Cloudy” to “Mobile Satellite Groundstation”.
• Reformatted the Units Conversion Table.



END OF DOCUMENT
