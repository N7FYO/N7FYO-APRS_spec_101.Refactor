\part{The Structure of this Specification}

This specification describes the overall requirements for developing software
that complies with APRS Protocol Version 1.0. The information flow starts
with the standard AX.25 UI-frame, and progresses downwards into more and
more detail as the use of each field in the frame is explored.
A key feature of the specification is the inclusion of dozens of detailed
examples of typical APRS packets and related math computations.
Here is an outline of the chapters:

% reduce baseline skip for this section

\paragraph{Introduction to APRS --}A brief background to APRS and a summary of its
main features.

\paragraph{The APRS Design Philosophy --}The fundamentals of APRS, highlighting
its use as a real-time tactical communications tool, the timing of
APRS transmissions and the use of generic digipeating.

\paragraph{APRS and AX.25 --}A brief refresher on the structure of the AX.25
UI-frame, with particular reference to the special ways in which APRS uses
the Destination and Source Address fields and the Information field.

\paragraph{APRS Data in the AX.25 Destination and Source Address Fields --}
Details of generic APRS callsigns and callsigns that specify display symbols
and APRS software version numbers. Also a summary of how Mic-E
encoded data is stored in the Destination Address field, and how the Source
Address SSID can specify a display icon.

\paragraph{APRS Data in the AX.25 Information Field --}Details of the principal
constituents of APRS data that are stored in the Information field. Contains
the APRS Data Type Identifiers table, and a summary of all the different
types of data that the Information field can hold.

\paragraph{Time and Position Formats --}Information on formats for timestamps,
latitude, longitude, position ambiguity, Maidenhead locators, NMEA data
and altitude.

\paragraph{APRS Data Extensions --}Details of optional data extensions for station
course/speed, wind speed/direction, power/height/gain, pre-calculated radio
range, DF signal strength and Area Object descriptor.

\paragraph{Position and DF Report Data Formats --}Full details of these report
formats.

\paragraph{Compressed Position Report Data Formats --}Full details of how station
position and APRS data extensions are compressed into very short packets.

\paragraph{Mic-E Data Format --}Mic-E encoding of station lat/long position, altitude,
course, speed, Mic-E message code, telemetry data and APRS digipeater
path into the AX.25 Destination Address and Information fields.

\paragraph{Object and Item Reports --}Full information on how to set up APRS
Objects and Items, and details of the encoding of Area Objects (circles, lines,
ellipses etc).

\paragraph{Weather Reports --}Full format details for weather reports from
standalone (positionless) weather stations and for reports containing
position information. Also details of storm data format.

\paragraph{Telemetry Data --}A description of the MIM/KPC-3+ telemetry data
format, with supporting information on how to tailor the interpretation of the
raw data to individual circumstances.

\paragraph{Messages, Bulletins and Announcements --}Full format information.

\paragraph{Station Capabilities, Queries and Responses --}Details of the ten different
types of query and expected responses.

\paragraph{Status Reports --}The format of general status messages, plus the special
cases of using a status report to contain meteor scatter beam heading/power
and Maidenhead locator.

\paragraph{Network Tunneling --}The use of the Source Path Header to allow
tunneling of APRS packets through third-party networks that do not
understand AX.25 addresses, and the use of the third-party Data Type
Identifier.

\paragraph{User-Defined Data Format --}APRS allows users to define their own data
formats for special purposes. This chapter describes how to do this.

\paragraph{Other Packets --} A general statement on how APRS is to handle any other
packet types that are not covered by this specification.

\paragraph{APRS Symbols --}How to specify APRS symbols and symbol overlays, in
position reports and in generic GPS destination callsigns.

\paragraph{APRS Data Formats --}An appendix containing all the APRS data formats
collected together for easy reference.

\paragraph{The APRS Symbol Tables --}A complete listing of all the symbols in the
Primary and Alternate Symbol Tables.

\paragraph{ASCII Code Table --}The full ASCII code, including decimal and hex
codes for each character (the decimal code is needed for compressed lat/long
and altitude computations), together with the hex codes for bit-shifted ASCII
characters in AX.25 addresses (useful for Mic-E decoding and general on-air
packet monitoring).

% Note: Add Unicode Here?

\paragraph{Glossary --}A handy one-stop reference for the many APRS-specific terms
used in this specification.

\paragraph{References --}Pointers to other documents that are relevant to this
specification.

% unchange baseline skip here
