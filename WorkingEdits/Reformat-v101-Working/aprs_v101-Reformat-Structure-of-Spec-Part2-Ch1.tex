\chapter {Chapter 1: Introduction to APRS}


\section{What is APRS?}

APRS is short for Automatic Position Reporting System, which was designed
by Bob Bruninga, WB4APR, and introduced by him at the 1992 TAPR/
ARRL Digital Communications Conference.

Fundamentally, APRS is a packet communications protocol for
disseminating live data to everyone on a network in real time. Its most visual
feature is the combination of packet radio with the Global Positioning
System (GPS) satellite network, enabling radio amateurs to automatically
display the positions of radio stations and other objects on maps on a PC.
Other features not directly related to position reporting are supported, such as
weather station reporting, direction finding and messaging.

APRS is different from regular packet in several ways:

\begin{itemize}

\item It provides maps and other data displays, for vehicle/personnel location
and weather reporting in real time.

\item It performs all communications using a one-to-many protocol, so that
everyone is updated immediately.

\item It uses generic digipeating, with well-known callsign aliases, so that prior
knowledge of network topology is not required.

\item It supports intelligent digipeating, with callsign substitution to reduce
network flooding.

\item Using AX.25 UI-frames, it supports two-way messaging and distribution
of bulletins and announcements, leading to fast dissemination of text
information.

\item It supports communications with the Kenwood TH-D7 and TM-D700
radios, which have built-in TNC and APRS firmware.

\end{itemize}

Conventional packet radio is really only useful for passing bulk message
traffic from point to point, and has traditionally been difficult to apply to
real-time events where information has a very short lifetime. APRS turns
packet radio into a real-time tactical communications and display system for
emergencies and public service applications.

APRS provides universal connectivity to all stations, but avoids the
complexity, time delays and limitations of a connected network. It permits
any number of stations to exchange data just like voice users would on a
voice net. Any station that has information to contribute simply sends it, and
all stations receive it and log it.

APRS recognizes that one of the greatest real-time needs at any special event
or emergency is the tracking of key assets. Where is the marathon leader?
Where are the emergency vehicles? What’s the weather at various points in
the county? Where are the power lines down? Where is the head of the
parade? Where is the mobile ATV camera? Where is the storm?
To address these questions, APRS provides a fully featured automatic
vehicle location and status reporting system. It can be used over any two-way
radio system including amateur radio, marine band, and cellular phone. There
is even an international live APRS tracking network on the Internet.


\section{APRS Features}

APRS runs on most platforms, including DOS, Windows 3.x, Windows
95/98, MacOS, Linux and Palm. Most implementations on these platforms
support the main features of APRS:

\begin{itemize}

\item \textbf{Maps --}APRS station positions can be plotted in real-time on maps,
with coverage from a few hundred yards to worldwide. Stations reporting
a course and speed are dead-reckoned to their present position. Overlay
databases of the locations of APRS digipeaters, US National Weather
Service sites and even amateur radio stores are available. It is possible to
zoom in to any point on the globe.


\item \textbf{Weather Station Reporting --} APRS supports the automatic display of
remote weather station information on the screen.

\item \textbf{DX Cluster Reporting --} APRS an ideal tool for the DX cluster user.
Small numbers of APRS stations connected to DX clusters can relay DX
station information to many other stations in the local area, reducing
overall packet load on the clusters.

\item \textbf{Internet Access --} The Internet can be used transparently to cross-link
local radio nets anywhere on the globe. It is possible to telnet into
Internet APRS servers and see hundreds of stations from all over the
world live. Everyone connected can feed their locally heard packets into
the APRS server system and everyone everywhere can see them.

\item \textbf{Messages --} Messages are two-way messages with acknowledgement.
All incoming messages alert the user on arrival and are held on the
message screen until killed.


\item \textbf{Bulletins and Announcements --} Bulletins and announcements are
addressed to everyone. Bulletins are sent a few times an hour for a few
hours, and announcements less frequently but possibly over a few days.

\item \textbf{Fixed Station Tracking --} In addition to automatically tracking mobile
GPS/LORAN-equipped stations, APRS also tracks from manual reports
or grid squares.

\item \textbf{Objects --} Any user can place an APRS Object on his own map, and
within seconds that object appears on all other station displays. This is
particularly useful for tracking assets or people that are not equipped
with trackers. Only one packet operator needs to know where things are
(e.g. by monitoring voice traffic), and as he maintains the positions and
movements of assets on his screen, all other stations running APRS will
display the same information.


\end{itemize}
